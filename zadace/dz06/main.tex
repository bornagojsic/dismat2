\documentclass{exam}
\usepackage[T1]{fontenc}
\usepackage{amsmath}
\usepackage{amssymb}
\usepackage{mathtools}
\usepackage{tikz}
\makeatletter
\newcommand\mathcircled[1]{%
  \mathpalette\@mathcircled{#1}%
}
\newcommand\@mathcircled[2]{%
  \tikz[baseline=(math.base)] \node[draw,circle,inner sep=1pt] (math) {$\m@th#1#2$};%
}
\usepackage[colorlinks=true, allcolors=blue]{hyperref}
\usepackage[makeroom]{cancel}
\usepackage[lmargin=71pt, tmargin=1.2in]{geometry}  %For centering solution box

\renewcommand{\thepartno}{\thequestion.\alph{partno}}
\renewcommand{\partlabel}{\alph{partno})}
\newcommand{\Mod}[1]{\ (\mathrm{mod}\ #1)}

\def \brojZadace {6}

\lhead{DISMAT 2 - Zadaća \brojZadace\\}
\rhead{Borna Gojšić\\}
% \chead{\hline} % Un-comment to draw line below header
\thispagestyle{empty}   %For removing header/footer from page 1

\begin{document}

\begingroup
\centering
\LARGE Diskretna matematika 2\\
\Large Zadaća \brojZadace\\
\large \today\\
\large Borna Gojšić\par
\endgroup
\rule{\textwidth}{0.4pt}
\pointsdroppedatright   %Self-explanatory
\printanswers
\renewcommand{\solutiontitle}{\noindent\textbf{Rj:}\enspace}   %Replace "Ans:" with starting keyword in solution box

\begin{questions}

\question Odredite sve kvadratne ostatke modulo 29.

\begin{solution}
  Reducirani sustav oznaka modulo 29 je $\{-14, \dots, -1, 1, \dots, 14\}$. Dakle, trebamo promatrati brojeve od 1 do 14 i pogledati ostatke njihovih kvadrata modulo 29 da bismo dobili sve kvadratne ostatke modulo 29.
  \vspace*{0.25cm}
  \newline
  \begin{tabular}{|c|c|c|c|c|c|c|c|c|c|c|c|c|c|c|}
    \hline
    $x$ & 1 & 2 & 3 & 4 & 5 & 6 & 7 & 8 & 9 & 10 & 11 & 12 & 13 & 14 \\
    \hline
    $x^2$ & 1 & 4 & 9 & 16 & 25 & 36 & 49 & 64 & 81 & 100 & 121 & 144 & 169 & 196 \\
    \hline
    $x^2 \Mod{29}$ & 1 & 4 & 9 & 16 & 25 & 7 & 20 & 6 & 23 & 13 & 5 & 28 & 24 & 22 \\
    \hline
  \end{tabular}
  \vspace*{0.25cm}
  \newline
  Dakle, kvadratni ostaci modulo 29 su 1, 4, 5, 6, 7, 9, 13, 16, 20, 22, 23, 24, 25 i 28.
\end{solution}

\question Ako je $p$ neparan prost broj, dokažite da je
\[
  \sum_{k=1}^{p-1} \left( \frac{k}{p} \right) = 0
\]

\begin{solution}
  Budući da u skupu $\{-\frac{p - 1}{2}, \dots, -1, 1, \dots, \frac{p - 1}{2}\}$ ima točno $\frac{p - 1}{2}$ brojeva koji su kvadratni ostaci modulo $p$ i svi ostali su kvadratni neostaci modulo $p$, imamo:
  \[
    \sum_{k=1}^{p-1} \left( \frac{k}{p} \right) = \frac{p - 1}{2} \cdot 1 + \frac{p - 1}{2} \cdot (-1) = 0
  \]
\end{solution}

\question Izračunajte Legendreove simbole
\begin{parts}
  \parbox{.1\linewidth}{
    \part $\displaystyle \left( \frac{-35}{97} \right)$
  }\hspace*{1cm}
  \parbox{.1\linewidth}{
    \part $\displaystyle \left( \frac{111}{991} \right)$
  }\hspace*{1cm}
  \parbox{.1\linewidth}{
    \part $\displaystyle \left( \frac{160}{163} \right)$
  }\hspace*{1cm}
  \parbox{.1\linewidth}{
    \part $\displaystyle \left( \frac{164}{167} \right)$
  }\hspace*{1cm}
  \parbox{.1\linewidth}{
    \part $\displaystyle \left( \frac{436}{683} \right)$
  }\hspace*{1cm}
\end{parts}

\begin{solution}
  \begin{parts}
    \part
    \begin{align*}
      \left( \frac{-35}{97} \right) &= \left( \frac{-1}{97} \right) \left( \frac{5}{97} \right) \left( \frac{7}{97} \right) = (-1)^{\frac{97 - 1}{2}} \cdot (-1)^{\frac{96 \cdot 4}{4}} \cdot \left( \frac{97}{5} \right) \cdot (-1)^{\frac{96 \cdot 6}{4}} \cdot \left( \frac{97}{7} \right)\\
      &= \left( \frac{2}{5} \right) \cdot \left( \frac{6}{7} \right) = -1 \cdot \left( \frac{-1}{7} \right) = -1 \cdot (-1)^{\frac{6}{2}} = 1
    \end{align*}

    \part
    \begin{align*}
      \left( \frac{111}{991} \right) &= \left( \frac{3}{991} \right) \left( \frac{37}{991} \right) = (-1)^{\frac{110 \cdot 990}{4}} \cdot \left( \frac{991}{3} \right) \cdot (-1)^{\frac{36 \cdot 990}{4}} \cdot \left( \frac{991}{37} \right)\\
      &= - \left( \frac{1}{3} \right) \left( \frac{29}{37} \right) = - (-1)^{\frac{28 \cdot 36}{4}} \cdot \left( \frac{37}{29} \right) = - \left( \frac{8}{29} \right) = - (1)^3 = -1
    \end{align*}

    \part
    \begin{align*}
      \left( \frac{160}{163} \right) &= \left( \frac{-3}{163} \right) = \left( \frac{-1}{163} \right) \left( \frac{3}{163} \right) = -1 \cdot (-1)^{\frac{2 \cdot 162}{4}} \left( \frac{163}{3} \right) = \left( \frac{1}{3} \right) = 1
    \end{align*}

    \part
    \begin{align*}
      \left( \frac{164}{167} \right) &= \left( \frac{-3}{167} \right) = \left( \frac{-1}{167} \right) \left( \frac{3}{167} \right) = -1 \cdot (-1)^{\frac{2 \cdot 166}{4}} \left( \frac{167}{3} \right) = \left( \frac{2}{3} \right) = -1
    \end{align*}

    \part
    \begin{align*}
      \left( \frac{436}{683} \right) &= \left( \frac{2}{683} \right)^2 \left( \frac{109}{683} \right) = (-1)^{\frac{108 \cdot 682}{4}} \left( \frac{683}{109} \right) = \left( \frac{29}{109} \right) = (-1)^{\frac{108 \cdot 28}{4}} \left( \frac{109}{29} \right)\\
      &= \left( \frac{-7}{29} \right) = \left( \frac{-1}{29} \right) \left( \frac{7}{29} \right) = (-1)^{\frac{6 \cdot 28}{4}} \left( \frac{29}{7} \right) = \left( \frac{1}{7} \right) = 1
    \end{align*}
  \end{parts}
\end{solution}

\question Izračunajte Jacobijeve simbole $\displaystyle \left( \frac{40}{403} \right)$ i $\displaystyle \left( \frac{907}{1455} \right)$

\begin{solution}
  Budući da je $403 = 13 \cdot 31$, imamo:
  \begin{align*}
    \left( \frac{40}{403} \right) &= \left( \frac{40}{13} \right) \left( \frac{40}{31} \right) = \left( \frac{1}{13} \right) \left( \frac{9}{31} \right) = 1
  \end{align*}
  Nadalje, budući da je $1455 = 3 \cdot 5 \cdot 97$, imamo:
  \begin{align*}
    \left( \frac{907}{1455} \right) &= \left( \frac{907}{3} \right) \left( \frac{906}{5} \right) \left( \frac{907}{97} \right) = \left( \frac{1}{3} \right) \left( \frac{2}{5} \right) \left( \frac{34}{97} \right) = -\left( \frac{2}{97} \right) \left( \frac{17}{97} \right) = - (-1)^{\frac{16 \cdot 96}{4}} \left( \frac{97}{17} \right)\\
    &= -\left( \frac{-5}{17} \right) = -\left( \frac{-1}{17} \right) \left( \frac{5}{17} \right) = - (-1)^{\frac{16}{2}} \cdot (-1)^{\frac{16 \cdot 4}{4}} \left( \frac{17}{5} \right) = - \left( \frac{2}{5} \right) = 1
  \end{align*}
\end{solution}

\pagebreak

\question
\begin{parts}
  \part Izračunajte Jacobijeve simbole $\displaystyle \left( \frac{-60}{377} \right)$ i $\displaystyle \left( \frac{-60}{323} \right)$
  \part Je li -60 kvadratni ostatak modulo 377? Detaljno obrazložite odgovor!
  \part Je li -60 kvadratni ostatak modulo 323? Detaljno obrazložite odgovor!
\end{parts}

\begin{solution}
  \begin{parts}
    \part Budući da je $377 = 13 \cdot 29$, imamo:
    \begin{align*}
      \left( \frac{-60}{377} \right) &= \left( \frac{-60}{13} \right) \left( \frac{-60}{29} \right) = \left( \frac{5}{13} \right) \left( \frac{-2}{29} \right) = \left[ (-1)^{\frac{48}{4}} \left( \frac{13}{5} \right) \right] \left[ (-1)^{\frac{28}{2}} \left( \frac{2}{29} \right) \right]\\
      &= \left( \frac{3}{5} \right) \cdot (-1) = 1
    \end{align*}
    Budući da je $323 = 17 \cdot 19$, imamo:
    \begin{align*}
      \left( \frac{-60}{323} \right) &= \left( \frac{-60}{17} \right) \left( \frac{-60}{19} \right) = \left[ \left( \frac{2}{17} \right) \right]^3 \left[ \left( \frac{2}{19} \right) \right]^4 = \left( \frac{2}{17} \right) = 1
    \end{align*}

    \part -60 nije kvadratni ostatak jer je $\displaystyle \left( \frac{-60}{13} \right) = -1$.

    \part Budući da je $\displaystyle \left( \frac{-60}{17} \right) = \left( \frac{-60}{19} \right) = 1$, znamo da je -60 kvadratni ostatak modulo 323.
  \end{parts}
\end{solution}

\question Odredite sve neparne proste brojeve $p$ takve da je:
\begin{parts}
  \part $\displaystyle \left( \frac{6}{p} \right) = 1$
  \part $\displaystyle \left( \frac{-60}{p} \right) = -1$
  \part $\displaystyle \left( \frac{40}{p} \right) = -1$
\end{parts}

\begin{solution}
  \begin{parts}
    \part
    \begin{align*}
      \left( \frac{6}{p} \right) &= \left( \frac{2}{p} \right) \left( \frac{3}{p} \right) = 1
    \end{align*}
    Imamo dvije mogućnosti:
    \begin{enumerate}
      \item Ako imamo da je $\left( \frac{2}{p} \right) = \left( \frac{3}{p} \right) = 1$, onda iz $\left( \frac{2}{p} \right) = 1$ slijedi $p \equiv 1, 7 \Mod{8}$, te $\left( \frac{3}{p} \right) = (-1)^{\frac{p-1}{2}} \left( \frac{p}{3} \right) = 1$. Dakle, ako je $p \equiv 1 \Mod{8}$, onda je i $p \equiv 1 \Mod{3}$, tj. $p \equiv 1 \Mod{24}$. A, ako je $p \equiv 7 \Mod{8}$, onda je i $p \equiv 2 \Mod{3}$, tj. $p \equiv 23 \Mod{24}$.

      \item Ako imamo da je $\left( \frac{2}{p} \right) = \left( \frac{3}{p} \right) = -1$ onda iz $\left( \frac{2}{p} \right) = -1$ slijedi $p \equiv 3, 5 \Mod{8}$, te $\left( \frac{3}{p} \right) = (-1)^{\frac{p-1}{2}} \left( \frac{p}{3} \right) = -1$. Dakle, ako je $p \equiv 3 \Mod{8}$, onda je i $p \equiv 1 \Mod{3}$, tj. $p \equiv 19 \Mod{24}$. A, ako je $p \equiv 5 \Mod{8}$, onda je i $p \equiv 2 \Mod{3}$, tj. $p \equiv 5 \Mod{24}$.
    \end{enumerate}
    Dakle, $p \equiv 1, 5, 19, 23 \Mod{24}$.

    \part
    \begin{align*}
      \left( \frac{-60}{p} \right) &= (-1)^{\frac{p-1}{2}} \left( \frac{2^2}{p} \right) \left( \frac{3}{p} \right) \left( \frac{5}{p} \right) = (-1)^{\frac{p-1}{2}} \cdot  (-1)^{\frac{2(p-1)}{4}} \cdot \left( \frac{p}{3} \right) \left( \frac{p}{5} \right)\\
      &= \left( \frac{p}{3} \right) \left( \frac{p}{5} \right) = -1
    \end{align*}
    Imamo dvije mogućnosti:
    \begin{enumerate}
      \item Ako je $\left( \frac{p}{3} \right) = 1$ i $\left( \frac{p}{5} \right) = -1$, onda imamo $p \equiv 1 \Mod{3}$ i $p \equiv \pm 2 \Mod{5}$. Dakle, $p \equiv 7, 13 \Mod{15}$.

      \item Ako je $\left( \frac{p}{3} \right) = -1$ i $\left( \frac{p}{5} \right) = 1$, onda imamo $p \equiv 2 \Mod{3}$ i $p \equiv \pm 1 \Mod{5}$. Dakle, $p \equiv 11, 14 \Mod{15}$.
    \end{enumerate}

    Dakle, imamo $p \equiv 7, 11, 13, 14 \Mod{15}$.

    \part
    \begin{align*}
      \left( \frac{40}{p} \right) &= \left( \frac{2^2}{p} \right)  \left( \frac{2}{p} \right) \left( \frac{5}{p} \right) = \left( \frac{2}{p} \right) \left( \frac{5}{p} \right) = -1
    \end{align*}

    Imamo dvije mogućnosti:
    \begin{enumerate}
      \item Ako je $\left( \frac{2}{p} \right) = 1$ i $\left( \frac{5}{p} \right) = -1$, onda imamo $p \equiv 1, 7 \Mod{8}$ i $p \equiv \pm 2 \Mod{5}$. Dakle, $p \equiv 7, 17, 23, 33 \Mod{40}$.

      \item Ako je $\left( \frac{2}{p} \right) = -1$ i $\left( \frac{5}{p} \right) = 1$, onda imamo $p \equiv 3, 5 \Mod{8}$ i $p \equiv \pm 1 \Mod{5}$. Dakle, $p \equiv 11, 19, 21, 29 \Mod{40}$
    \end{enumerate}

    Dakle, imamo $p \equiv 7, 11, 17, 19, 21, 23, 29, 33 \Mod{40}$.
  \end{parts}
\end{solution}

\question
\begin{parts}
  \part Odredite sve neparne proste brojeve $p$ takve da je $\displaystyle \left( \frac{-3}{p} \right) = 1$
  \part Dokažite da postoji beskonačno mnogo prostih brojeva oblika $6k + 1$.
\end{parts}

\begin{solution}
  \begin{parts}
    \part
    \[
      \left( \frac{-3}{p} \right) = \left( \frac{-1}{p} \right) \left( \frac{3}{p} \right) = (-1)^{\frac{p-1}{2}} \cdot (-1)^{\frac{(p-1) \cdot 2}{4}} \left( \frac{p}{3} \right) = \left( \frac{p}{3} \right) = 1
    \]
    Jedini kvadratni ostatak modulo 3 je 1, pa imamo $p \equiv 1 \Mod{3}$, tj. $p = 3k + 1$. Budući da je $p$ neparan, imamo i $p = 6k + 1$.
    \part Neka su $p_1, p_2, \dots, p_n$ svi prosti brojevi oblika $6k + 1$. Promotrimo broj $m = p_1^2 p_2^2 \cdots p_n^2 + 3$.
    \[
      m \equiv 0 \Mod{p} \iff x^2 \equiv -3 \Mod{p} \iff \left( \frac{-3}{p} \right) = 1
    \]
    Dakle, $m$ ima prosti faktor $p$ oblika $6k + 1$. Očito $p \neq p_i$ za $i \in \{1, 2, \dots, n\}$, pa imamo kontradikciju. Dakle, postoji beskonačno mnogo prostih brojeva oblika $6k + 1$.
  \end{parts}
\end{solution}

\question Izračunajte
\begin{parts}
  \part $\displaystyle \left( \frac{17}{p} \right)$
  \part $\displaystyle \left( \frac{19}{p} \right)$
\end{parts}
za sve neparne proste brojeve $p$.

\begin{solution}
  \begin{parts}
    \part
    \[
      \left( \frac{17}{p} \right) = (-1)^{\frac{(p-1) \cdot 16}{4}} \left( \frac{p}{17} \right) = \left( \frac{p}{17} \right)
    \]

    Sada ćemo izračunati sve kvadratne ostatke modulo 17:

    \begin{tabular}{|c|c|c|c|c|c|c|c|c|}
      \hline
      $x$ & 1 & 2 & 3 & 4 & 5 & 6 & 7 & 8\\
      \hline
      $x^2$ & 1 & 4 & 9 & 16 & 25 & 36 & 49 & 64\\
      \hline
      $x^2 \Mod{29}$ & 1 & 4 & 9 & 16 & 8 & 2 & 15 & 13\\
      \hline
    \end{tabular}

    Dakle, imamo $\displaystyle \left( \frac{17}{p} \right) =
    \begin{cases}
      1, & \text{ako je } p \equiv 1, 2, 4, 8, 9, 13, 15, 16 \Mod{17}\\
      0, & \text{ako je } p = 17\\
      -1, & \text{ako je } p \equiv 3, 5, 6, 7, 10, 11, 12, 14 \Mod{17}
    \end{cases}$

    \part
    \[
      \left( \frac{19}{p} \right) = (-1)^{\frac{(p-1) \cdot 18}{4}} \left( \frac{p}{19} \right) = (-1)^{\frac{p-1}{2}} \left( \frac{p}{19} \right)
    \]

    Sada ćemo izračunati sve kvadratne ostatke modulo 19:

    \begin{tabular}{|c|c|c|c|c|c|c|c|c|c|}
      \hline
      $x$ & 1 & 2 & 3 & 4 & 5 & 6 & 7 & 8 & 9\\
      \hline
      $x^2$ & 1 & 4 & 9 & 16 & 25 & 36 & 49 & 64 & 81\\
      \hline
      $x^2 \Mod{29}$ & 1 & 4 & 9 & 16 & 6 & 17 & 11 & 7 & 5\\
      \hline
    \end{tabular}

    Imamo 4 slučaja, ali možemo promotriti samo prva dva i ostale dobiti komplementom:
    \begin{enumerate}
      \item Ako je $\displaystyle \left( \frac{p}{19} \right) = 1$ i $p \equiv 1 \Mod{4}$, onda je $p \equiv 1, 4, 5, 6, 7, 11, 13, 16, 17 \Mod{19}$ i $p \equiv 1 \Mod{4}$. Dakle, $p \equiv 1, 5, 13, 17, 25, 45, 49, 61, 73 \Mod{76}$.
      \item Ako je $\displaystyle \left( \frac{p}{19} \right) = -1$ i $p \equiv 3 \Mod{4}$ onda je $p \equiv 2, 3, 8, 9, 10, 12, 14, 15, 18 \Mod{19}$ i $p \equiv 3 \Mod{4}$. Dakle, $p \equiv 3, 15, 27, 31, 47, 59, 67, 71, 75 \Mod{76}$.
    \end{enumerate}
    Dakle, imamo:
    \[
      \left( \frac{19}{p} \right) =
      \begin{cases}
        1, &\text{ako je } p \equiv 1, 3, 5, 13, 15, 17, 25, 27, 31, 45,\\
        &\qquad \qquad \ \ \, 47, 49, 59, 61, 67, 71, 73, 75 \Mod{76}\\
        0, & \text{ako je } p = 19\\
        -1, & \text{inače}
      \end{cases}
    \]

  \end{parts}
\end{solution}

\pagebreak

\question Odredite sve neparne proste brojeve $p$ takve da kongruencija $x^2 + 45 \equiv 0 \Mod{p}$ ima rješenja.

\begin{solution}
  \begin{align*}
    \left( \frac{-45}{p} \right) = \left( \frac{-1}{p} \right) \left( \frac{5}{p} \right) \left( \frac{3^2}{p} \right) = (-1)^{\frac{p-1}{2}} \cdot (-1)^{\frac{4(p-1)}{4}} \left( \frac{p}{5} \right) = (-1)^{\frac{p-1}{2}} \left( \frac{p}{5} \right) = 1
  \end{align*}
  Imamo dvije mogućnosti:
  \begin{enumerate}
    \item Kvadratni ostaci modulo 5 su 1 i 4 pa kongruencija ima rješenja za $p \equiv \pm 1 \Mod{5}$ i $p \equiv 1 \Mod{4}$, tj. $p \equiv 1, 9 \Mod{20}$.
    \item Kvadratni neostaci modulo 5 su 2 i 3 pa kongruencija ima rješenja za $p \equiv \pm 2 \Mod{5}$ i $p \equiv 3 \Mod{4}$, tj. $p \equiv 3, 7 \Mod{20}$.
  \end{enumerate}
\end{solution}

\question Neka je $p$ neparan prost broj s primitivnim korijenom $g$ te neka je $a$ cijeli broj takav da je $\text{nzd}(a, p) = 1$. Dokažite da je $a$ kvadratni ostatak modulo $p$ ako i samo ako je indeks $\text{ind}_g a$ paran.

\begin{solution}
  Neka je $g$ primitvni korijen modulo $p$. Pretpostavimo prvo da je $\text{ind}_g a$ paran. Tada imamo $g^{2k} \equiv a \Mod{p}$ za neki $k \in \mathbb{Z}$. Dakle, $a$ je kvadratni ostatak modulo $p$. Pretpostavimo sada da je $a$ kvadratni ostatak modulo $p$. Tada imamo $x^2 \equiv a \Mod{p}$ za neki $x \in \mathbb{Z}$. Po definciji primitvnog korijena, postoji $k \in \mathbb{Z}$ takav da je $g^k \equiv x \Mod{p}$. Dakle, $g^{2k} \equiv a \Mod{p}$, tj. $\text{ind}_g a$ je paran.
\end{solution}

\question Neka je $q$ prost broj oblika $q = p^2 + 4a^2$ gdje je $p$ neparan prost broj te $a$ cijeli broj. Dokažite da je
$\displaystyle \left( \frac{p}{q} \right) = 1$. \newline
\underline{Uputa}: Koristite Gaussov zakon reciprociteta

\begin{solution}
  Budući da je $q = p^2 + 4a^2$, i $p$ je neparan prost broj, imamo $q \equiv (\pm 1)^2 \equiv 1 \Mod{4}$. Dakle, $q$ je prost broj oblika $4k + 1$. Nadalje,
  \[
    \left( \frac{q}{p} \right) = \left( \frac{p^2 + 4a^2}{p} \right) = \left( \frac{(2a)^2}{p} \right) = 1
  \]
  Sada koristimo Gaussov zakon reciprociteta:
  \[
    \left( \frac{p}{q} \right) \cdot 1 = \left( \frac{p}{q} \right) \left( \frac{q}{p} \right) = (-1)^{\frac{(p-1)(q-1)}{4}} = (-1)^{\frac{(p-1) \cdot 4k}{4}} = 1
  \]
\end{solution}

\pagebreak

\question Neka je $a$ neparan prost broj te neka je $b$ cijeli broj takav da je $p = a^2 + 5b^2$ prost. Dokažite da je $a$ kvadratni ostatak modulo $p$ ako i samo ako je $p \equiv 1 \Mod{5}$.

\begin{solution}
  \[
    \left( \frac{a}{p} \right) = (-1)^{\frac{(a-1)(p-1)}{4}} \cdot \left( \frac{p}{a} \right)
  \]
  Budući da je $a$ neparan prost broj, imamo $a \equiv \pm 1 \Mod{4}$ pa je $p \equiv 1 + 0 \equiv 1 \Mod{4}$ jer $b$ mora biti paran kako bi $p$ bio neparan. Dakle, imamo:
  \[
    \left( \frac{a}{p} \right) = (-1)^{\frac{(a-1)(p-1)}{4}} \cdot \left( \frac{p}{a} \right) = \left( \frac{p}{a} \right)
  \]
  Znamo da je
  \[
    \left( \frac{p}{a} \right) \equiv (a^2 + 5b^2)^{\frac{a-1}{2}} \equiv 5^{\frac{a-1}{2}} b^{a-1} \equiv 5^{\frac{a-1}{2}} \equiv \left( \frac{5}{a} \right) \Mod{a}
  \]
  Dakle, imamo
  \[
    \left( \frac{p}{a} \right) = \left( \frac{5}{a} \right) = (-1)^{a-1} \cdot \left( \frac{a}{5} \right) = \left( \frac{a}{5} \right)
  \]
  Ako je $a$ kvadratni ostatak modulo 5, onda je $a = 5k \pm 1$ pa je $p = 5l + 1$. Dakle,
  \[
    \left( \frac{a}{p} \right) = \left( \frac{p}{a} \right) = \left( \frac{5}{a} \right) = 1
  \]
  Ako je $a$ kvadratni neostatak modulo 5, onda je $a = 5k \pm 2$ pa je $p = 5l + 4$.
  \[
    \left( \frac{a}{p} \right) = \left( \frac{p}{a} \right) = \left( \frac{5}{a} \right) = -1
  \]
  Jedini preostali slučaj je $a = 5$, ali onda imamo $25 \mid p$ pa $p$ nije prost. Dakle, $a$ je kvadratni ostatak modulo $p$ ako i samo ako je $p \equiv 1 \Mod{5}$.
\end{solution}

\pagebreak

\question Riješite sustav kongruencija
\begin{align*}
  x^2 &\equiv 21 \Mod{67} \\
  x^2 &\equiv 44 \Mod{83}
\end{align*}
\underline{Uputa}: Uočite $67 \equiv 83 \equiv 3 \Mod{4}$.

\begin{solution}
  Budući da su $67$ i $83$ oba oblika $4k + 3$, imamo:
  \begin{align*}
    x &\equiv \pm 21^{\frac{67 + 1}{4}} \equiv \pm 21^{17} \equiv \pm 21 \cdot 39^8 \equiv \pm 21 \cdot (-20)^4 \equiv \pm 21 \cdot (-2)^2 \equiv \pm 17 \Mod{67} \\
    x &\equiv \pm 44^{\frac{83 + 1}{4}} \equiv \pm 44^{21} \equiv \pm 44 \cdot 27^{10} \equiv \pm 44 \cdot 65^5 \equiv \pm 44 \cdot 65 \cdot (-8)^2 \equiv \pm 38 \cdot 64 \equiv \pm 25
    \Mod{83}
  \end{align*}
  Sada možemo riješiti 4 implicitna sustava kongruencija malim kineskim teoremom:
  \vspace*{0.25cm}
  \newline
  \begin{tabular}{|c|c|c|c|c|c|c|}
    \hline
    & & & & & &\\[-1em]
    $x$ & $y$ & $g$ & $u$ & $v$ & $w$ & $\left\lfloor \frac{g}{w} \right\rfloor$\\
    & & & & & &\\[-1em]
    \hline
    1 & 0 & 83 & 0 & 1 & 67 & 0\\
    0 & 1 & 67 & 1 & -1 & 16 & 4\\
    1 & -1 & 16 & -4 & 5 & 3 & 5\\
    -4 & 5 & 3 & 21 & -26 & 1 & 3\\
    21 & -26 & 1 & & & 0 & \\
    \hline
  \end{tabular}
  \vspace*{0.25cm}
  \newline
  Dakle, $21 \cdot 83 - 26 \cdot 67 = 1$ pa imamo $x_1^+ = 21 \cdot 83 \cdot 17 - 26 \cdot 67 \cdot 25 \equiv 2764 \Mod{5561}$ i $x_2^+ = 21 \cdot 83 \cdot 17 + 26 \cdot 67 \cdot 25 \equiv 888 \Mod{5561}$. Dakle, rješenja su $x \equiv \pm 2764, \pm 888 \Mod{5561}$.
\end{solution}

\end{questions}

\end{document}

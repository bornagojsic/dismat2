\documentclass{exam}
\usepackage[T1]{fontenc}
\usepackage{amsmath}
\usepackage{amssymb}
\usepackage{mathtools}
\usepackage{tikz}
\makeatletter
\newcommand\mathcircled[1]{%
  \mathpalette\@mathcircled{#1}%
}
\newcommand\@mathcircled[2]{%
  \tikz[baseline=(math.base)] \node[draw,circle,inner sep=1pt] (math) {$\m@th#1#2$};%
}
\usepackage[colorlinks=true, allcolors=blue]{hyperref}
\usepackage[makeroom]{cancel}
\usepackage[lmargin=71pt, tmargin=1.2in]{geometry}  %For centering solution box

\renewcommand{\thepartno}{\thequestion.\alph{partno}}
\renewcommand{\partlabel}{\alph{partno})}
\newcommand{\Mod}[1]{\ (\mathrm{mod}\ #1)}

\def \brojZadace {11}

\lhead{DISMAT 2 - Zadaća \brojZadace\\}
\rhead{Borna Gojšić\\}
% \chead{\hline} % Un-comment to draw line below header
\thispagestyle{empty}   %For removing header/footer from page 1

\begin{document}

\begingroup
\centering
\LARGE Diskretna matematika 2\\
\Large Zadaća \brojZadace\\
\large \today\\
\large Borna Gojšić\par
\endgroup
\rule{\textwidth}{0.4pt}
\pointsdroppedatright   %Self-explanatory
\printanswers
\renewcommand{\solutiontitle}{\noindent\textbf{Rj:}\enspace}   %Replace "Ans:" with starting keyword in solution box

\begin{questions}

\question U RSA kriptosustavu s javnim ključem $(n, e)$, gdje je $n = 86267 = 281 \cdot 307$ i $e = 65537$, šifrirajte otvoreni tekst $x = 1245$. Odredite pripadni tajni ključ $d$.

\begin{solution}
  Znamo da je $1245 = 3 \cdot 5 \cdot 83$. Dakle, imamo
  \begin{align*}
    1245^e &\equiv 1245^{65537} \equiv 1245^{2^{16}} \cdot 1245 \equiv (-2781)^{2^{15}} \cdot 1245 \Mod{n}\\
    &\equiv (-30069)^{2^{14}} \cdot 1245 \equiv (-19666)^{2^{13}} \cdot 1245 \equiv (16595)^{2^{12}} \cdot 1245 \Mod{n}\\
    &\equiv (29761)^{2^{11}} \cdot 1245 \equiv (13832)^{2^{10}} \cdot 1245 \equiv (-15982)^{2^9} \cdot 1245 \Mod{n}\\
    &\equiv (-12263)^{2^8} \cdot 1245 \equiv (17788)^{2^7} \cdot 1245 \equiv (-14412)^{2^6} \cdot 1245 \Mod{n}\\
    &\equiv (-25192)^{2^5} \cdot 1245 \equiv 56812^{2^4} \cdot 1245 \equiv 9806^{2^3} \cdot 1245 \Mod{n}\\
    &\equiv 56198^{2^2} \cdot 1245 \equiv (-19666)^2 \cdot 1245 \equiv 16595 \cdot 1245 \equiv 42962 \Mod{n}.
  \end{align*}
  Dakle, šifrat je $y = 42962$.
  Znamo da je $\varphi(n) = (p - 1)(q - 1) = 280 \cdot 306 = 2^3 \cdot 5 \cdot 7 \cdot 2 \cdot 3^2 \cdot 17 = 2^4 \cdot 3^2 \cdot 5 \cdot 7 \cdot 17 = 85680$.
  Dakle, imamo
  \begin{align*}
    65537 d &\equiv 1 \Mod{85680} \Rightarrow 65537 d = 85680 k + 1 \Rightarrow 65537 d - 85680 k = 1.
  \end{align*}
  Dakle, ovu kongruenciju možemo riješiti proširenim Euklidovim algoritmom.

  \begin{tabular}{|c|c|c|c|c|c|c|}
    \hline
    & & & & & &\\[-1em]
    $x$ & $y$ & $g$ & $u$ & $v$ & $w$ & $\left\lfloor \frac{g}{w} \right\rfloor$\\
    & & & & & &\\[-1em]
    \hline
    1 & 0 & 85680 & 0 & 1 & 65537 & 1\\
    0 & 1 & 65537 & 1 & -1 & 20143 & 3\\
    1 & -1 & 20143 & -3 & 4 & 5108 & 3\\
    -3 & 4 & 5108 & 10 & -13 & 4819 & 1\\
    10 & -13 & 4819 & -13 & 17 & 289 & 16\\
    -13 & 17 & 289 & 218 & -285 & 195 & 1\\
    218 & -285 & 195 & -231 & 302 & 94 & 2\\
    -231 & 302 & 94 & 680 & -889 & 7 & 13\\
    680 & -889 & 7 & -9071 & 11859 & 3 & 2\\
    -9071 & 11859 & 3 & 18822 & -24607 & 1 & 3\\
    18822 & -24607 & 1 & & & 0 &\\
    \hline
  \end{tabular}

  Dakle, $85680 \cdot (18822) + 65537 \cdot (-24607) = 1$ pa je $d \equiv -24607 \equiv 61073 \Mod{85680}$.
\end{solution}

\pagebreak

\question U nekoj banci se za šifriranje troznamenkastih PIN-ova koristi RSA kriptosustav s javnim ključem $(n, e)$, gdje je $n = 1411 = 17 \cdot 83$ i $e = 835$. Koji PIN ima Alice ako je šifrat njezinog PIN-a $002$?

\begin{solution}
  Budući da je $n = 17 \cdot 83 = 1411$, imamo $\varphi(n) = (17 - 1)(83 - 1) = 16 \cdot 82 = 1312$. Dakle, tražimo $d$ takav da vrijedi $835d \equiv 1 \Mod{1312}$, tj. $835d - 1312k = 1$.

  \begin{tabular}{|c|c|c|c|c|c|c|}
    \hline
    & & & & & &\\[-1em]
    $x$ & $y$ & $g$ & $u$ & $v$ & $w$ & $\left\lfloor \frac{g}{w} \right\rfloor$\\
    & & & & & &\\[-1em]
    \hline
    1 & 0 & 1312 & 0 & 1 & 835 & 1\\
    0 & 1 & 835 & 1 & -1 & 477 & 1\\
    1 & -1 & 477 & -1 & 2 & 358 & 1\\
    -1 & 2 & 358 & 2 & -3 & 119 & 3\\
    2 & -3 & 119 & -7 & 11 & 1 & 119\\
    -7 & 11 & 1 & & & 0 &\\
    \hline
  \end{tabular}

  Dakle, $1312 \cdot (-7) + 835 \cdot (11) = 1$ pa je $d = 11$. Dakle, $002^{11} \equiv 2^{11} \equiv 2048 \equiv 637 \Mod{1411}$.
\end{solution}

\question Alice je poslala istu poruku $m$ nekolicini agenata. Eva je presrela šifrate $c_1, c_2, c_3$ za trojicu agenata čiji su javni ključevi $n_1, n_2$ i $n_3$. Poznato je da Alice i agenti koriste RSA kriptosustav s javnim eksponentom $e = 3$. Za zadane
\begin{align*}
  n_1 &= 217, \ c_1 = 153, \\
  n_2 &= 299, \ c_2 = 226, \\
  n_3 &= 319, \ c_3 = 298,
\end{align*}
pokažite kako će Eva otkriti poruku $m$ (bez poznavanja faktorizacije modula $n_1, n_2, n_3$).

\begin{solution}
  Imamo sustav kongruencija:
  \begin{align*}
    x = m^3 \equiv 153 \Mod{217}, \quad x = m^3 \equiv 226 \Mod{299}, \quad x = m^3 \equiv 298 \Mod{319}.
  \end{align*}
  Dakle, imamo $M = 217 \cdot 299 \cdot 319 = 20697677$ i $x_0 = 299 \cdot 319 x_1 + 217 \cdot 319 x_2 + 217 \cdot 299 x_3$.

  Sada rješavamo kongruencije:
  \begin{align*}
    299 \cdot 319 \cdot x_1 &\equiv 118 \Mod{217} \implies 118 \cdot x_1 \equiv 153 \Mod{217}\\
    217 \cdot 319 \cdot x_2 &\equiv 154 \Mod{299} \implies 154 \cdot x_2 \equiv 226 \Mod{299}\\
    217 \cdot 299 \cdot x_3 &\equiv 298 \Mod{319} \implies 126 \cdot x_3 \equiv 298 \Mod{319}.
  \end{align*}

  Dobivamo $x_1 = 176$, $x_2 = 17$, $x_3 = 53$ pa je $x_0 = 299 \cdot 319 \cdot 176 + 217 \cdot 319 \cdot 17 + 217 \cdot 299 \cdot 53 = 21402646 \equiv 704969 \Mod{M}$. Dakle, $m = \sqrt[3]{704969} = 89$.
\end{solution}

\pagebreak

\question Alice je poslala istu poruku $m$ nekolicini agenata. Eva je presrela šifrate $c_1, c_2, c_3$ za trojicu agenata čiji su javni ključevi $n_1, n_2$ i $n_3$. Poznato je da Alice i agenti koriste RSA kriptosustav s javnim eksponentom $e = 3$. Za zadane
\begin{align*}
  n_1 &= 161, \ c_1 = 57, \\
  n_2 &= 247, \ c_2 = 96, \\
  n_3 &= 493, \ c_3 = 272,
\end{align*}
pokažite kako će Eva otkriti poruku $m$ (bez poznavanja faktorizacije modula $n_1, n_2, n_3$).

\begin{solution}
  Imamo sustav kongruencija:
  \begin{align*}
    x = m^3 \equiv 57 \Mod{161}, \quad x = m^3 \equiv 96 \Mod{247}, \quad x = m^3 \equiv 272 \Mod{493}.
  \end{align*}
  Dakle, imamo $M = 161 \cdot 247 \cdot 493 = 19605131$ i $x_0 = 247 \cdot 493 x_1 + 161 \cdot 493 x_2 + 161 \cdot 247 x_3$.
  Sada rješavamo kongruencije:

  \begin{align*}
    247 \cdot 493 \cdot x_1 &\equiv 57 \Mod{161} \implies 55 \cdot x_1 \equiv 57 \Mod{161}\\
    161 \cdot 493 \cdot x_2 &\equiv 96 \Mod{247} \implies 86 \cdot x_2 \equiv 96 \Mod{247}\\
    161 \cdot 247 \cdot x_3 &\equiv 272 \Mod{493} \implies 327 \cdot x_3 \equiv 272 \Mod{493}.
  \end{align*}
  Dobivamo $x_1 = 83, x_2 = 116, x_3 = 34$ pa je $x_0 = 247 \cdot 493 \cdot 83 + 161 \cdot 493 \cdot 116 + 161 \cdot 247 \cdot 34 = 20666339 \equiv 1061208 \Mod{M}$. Dakle, $m = \sqrt[3]{1061208} = 102$.
\end{solution}

\pagebreak

\question Zadan je RSA kriptosustav s javnim ključem $(n, e) = (69627997, 43206989)$. Odredite pomoću Wienerovog napada skup mogućih tajnih ključeva $d$ ako je poznato da vrijedi $d < \frac{1}{3} \sqrt[4]{n}$.

\begin{solution}
  Winerov napad počinjemo zapisivanjem $\frac{e}{n}$ u obliku verižnog razlomka. Dakle, imamo:
  \begin{align*}
    69627997 &= 1 \cdot 43206989 + 26421008\\
    43206989 &= 1 \cdot 26421008 + 16785981\\
    26421008 &= 1 \cdot 16785981 + 9635027\\
    16785981 &= 1 \cdot 9635027 + 7150954\\
    9635027 &= 1 \cdot 7150954 + 2484073\\
    7150954 &= 2 \cdot 2484073 + 2182808\\
    2484073 &= 1 \cdot 2182808 + 301265\\
    2182808 &= 7 \cdot 301265 + 73953\\
    301265 &= 4 \cdot 73953 + 5453\\
    73953 &= 13 \cdot 5453 + 3064\\
    5453 &= 1 \cdot 3064 + 2389\\
    3064 &= 1 \cdot 2389 + 675\\
    2389 &= 3 \cdot 675 + 364\\
    675 &= 1 \cdot 364 + 311\\
    364 &= 1 \cdot 311 + 53\\
    311 &= 5 \cdot 53 + 46\\
    53 &= 1 \cdot 46 + 7\\
    46 &= 6 \cdot 7 + 4\\
    7 &= 1 \cdot 4 + 3\\
    4 &= 1 \cdot 3 + 1\\
    3 &= 3 \cdot 1 + 0
  \end{align*}
  Dakle, $\frac{43206989}{69627997} = [0; 1, 1, 1, 1, 1, 2, 1, 7, 4, 13, 1, 4, 3, 1, 5, 1, 6, 1]$. Sada znamo da je $d$ nazivnik neke od konvergenti za koji je $d < \frac{1}{3} \sqrt[4]{n} < 31$. Dakle, imamo $q_{-1} = 0, q_{0} = 1$ i rekurziju $q_i = q_{i-1} a_i + q_{i-2}$ za $i \geq 1$.

  \begin{tabular}{|c|c|c|c|c|c|c|c|c|c|c|}
    \hline
    $i$ & -1 & 0 & 1 & 2 & 3 & 4 & 5 & 6 & 7 & 8\\
    \hline
    $a_i$ & & 0 & 1 & 1 & 1 & 1 & 1 & 2 & 1 & 7\\
    \hline
    $q_i$ & 0 & 1 & 1 & 2 & 3 & 5 & 8 & 21 & 29 & 224\\
    \hline
  \end{tabular}

  Dakle, $d \in \{1, 2, 3, 5, 8, 21, 29\}$.
\end{solution}

\pagebreak

\question Zadan je RSA kriptosustav s javnim ključem $(n, e) = (60677801, 47474687)$. Odredite pomoću Wienerovog napada skup mogućih tajnih ključeva $d$ ako je poznato da vrijedi $d < \frac{1}{3} \sqrt[4]{n}$.

\begin{solution}
  Wienerov napad počinjemo zapisivanjem $\frac{e}{n}$ u obliku verižnog razlomka. Dakle, imamo:
  \begin{align*}
    60677801 &= 1 \cdot 47474687 + 13203114\\
    47474687 &= 3 \cdot 13203114 + 7865345\\
    13203114 &= 1 \cdot 7865345 + 5337769\\
    7865345 &= 1 \cdot 5337769 + 2527576\\
    5337769 &= 2 \cdot 2527576 + 282617\\
    2527576 &= 8 \cdot 282617 + 266640\\
    282617 &= 1 \cdot 266640 + 15977\\
    266640 &= 16 \cdot 15977 + 11008\\
    15977 &= 1 \cdot 11008 + 4969\\
    11008 &= 2 \cdot 4969 + 1070\\
    4969 &= 4 \cdot 1070 + 689\\
    1070 &= 1 \cdot 689 + 381\\
    689 &= 1 \cdot 381 + 308\\
    381 &= 1 \cdot 308 + 73\\
    308 &= 4 \cdot 73 + 16\\
    73 &= 4 \cdot 16 + 9\\
    16 &= 1 \cdot 9 + 7\\
    9 &= 1 \cdot 7 + 2\\
    7 &= 3 \cdot 2 + 1\\
    2 &= 2 \cdot 1 + 0
  \end{align*}
  Dakle, $\frac{47474687}{60677801} = [0; 1, 3, 1, 1, 2, 8, 1, 16, 1, 2, 4, 1, 1, 1, 4, 4, 1, 1, 3, 2]$. Sada znamo da je $d$ nazivnik neke od konvergenti za koji je $d < \frac{1}{3} \sqrt[4]{n} < 30$. Dakle, imamo $q_{-1} = 0, q_{0} = 1$ i rekurziju $q_i = q_{i-1} a_i + q_{i-2}$ za $i \geq 1$.

  \begin{tabular}{|c|c|c|c|c|c|c|c|c|c|c|}
    \hline
    $i$ & -1 & 0 & 1 & 2 & 3 & 4 & 5 & 6\\
    \hline
    $a_i$ & & 0 & 1 & 3 & 1 & 1 & 2 & 8\\
    \hline
    $q_i$ & 0 & 1 & 1 & 4 & 5 & 9 & 23 & 193\\
    \hline
  \end{tabular}

  Dakle, $d \in \{1, 4, 5, 9, 23\}$.
\end{solution}

\question U RSA kriptosustavu je $n = pq = 51809$, gdje su $p$ i $q$ prosti brojevi. Špijuniranjem ste otkrili da je $\sigma(n) = 52416$ ($\sigma(n)$ je suma djelitelja broja $n$). Odredite $p$ i $q$ bez poznavanja faktorizacije od $n$.
\underline{Uputa:} Iskoristite multiplikativnost od $\sigma$ i Vièteove formule.

\begin{solution}
  Budući da je
  \begin{align*}
    \sigma(n) = \sigma(pq) = \sigma(p) \sigma(q) = (p + 1)(q + 1) = 1 + p + q + pq = 1 + p + q + n
  \end{align*}
  Imamo $p + q = \sigma(n) - n - 1 = 52416 - 51809 - 1 = 606$ te znamo $pq = 51809$. Dakle, $p$ i $q$ su rješenja kvadratne jednadžbe $x^2 - 606x + 51809 = 0$. Rješenja su $p = 103$ i $q = 503$.
\end{solution}

\question U Rabinovom kriptosustavu s parametrima
\begin{align*}
  (n, p, q) &= (2773, 47, 59),
\end{align*}
dešifrirajte šifrat $y = 2729$. Poznato je da je otvoreni tekst prirodan broj $x < n$ kojem su zadnja četiri bita u binarnom zapisu međusobno jednaka.

\begin{solution}
  Najprije trebamo naći kvadratne korijene od $y$ modulo $p$ i modulo $q$. Budući da je $p \equiv q \equiv 3 \Mod{4}$, imamo
  \begin{align*}
    x &= \pm \sqrt{y} \Mod{p} = \pm \sqrt{2729} \Mod{47} \equiv \pm 2729^{12} \equiv \pm 3^{12} \Mod{47}\\
    x &= \pm \sqrt{y} \Mod{q} = \pm \sqrt{2729} \Mod{59} \equiv 2729^{15} \equiv \pm 15^{15} \Mod{59}
  \end{align*}
  Nadalje, imamo:
  \begin{align*}
    x \equiv \pm 9^6 \equiv \pm 81^3 \equiv \pm 34^3 \equiv \pm 28 \cdot 34 \equiv \pm 12 \Mod{47},\\
    x \equiv \pm 15^{14} \cdot 15 \equiv \pm 48^{7} \cdot 15 \equiv \pm 48^{6} \cdot 12 \equiv \pm 3^3 \cdot 12 \equiv \pm 29 \Mod{59}.
  \end{align*}
  Sada rješavamo ovaj sustav od 4 kongruencije malim kineskim teoremom. Dakle, trebamo naći $u$ i $v$ takve da je $59u + 47v = 1$. Te ćemo brojeve naći proširenim Euklidovim algoritmom.

  \begin{tabular}{|c|c|c|c|c|c|c|}
    \hline
    & & & & & &\\[-1em]
    $x$ & $y$ & $g$ & $u$ & $v$ & $w$ & $\left\lfloor \frac{g}{w} \right\rfloor$\\
    & & & & & &\\[-1em]
    \hline
    1 & 0 & 59 & 0 & 1 & 47 & 1\\
    0 & 1 & 47 & 1 & -1 & 12 & 3\\
    1 & -1 & 12 & -3 & 4 & 11 & 1\\
    -3 & 4 & 11 & 4 & -5 & 1 & 11\\
    4 & -5 & 1 & & & 0 &\\
    \hline
  \end{tabular}

  Dakle, $u = 4$ i $v = -5$. Sada su rješenja sustava kongruencija dana s $x \equiv 59 \cdot 4 \cdot (\pm 12) + 47 \cdot (-5) \cdot (\pm 29) \Mod{2773}$. Dakle, $x \equiv \pm 1563, \pm 1328 \Mod{2773}$, tj. imamo
  \begin{align*}
    x_1 &= 1563 = 11000011011_2\\
    x_2 &= 1210 = 10010111010_2\\
    x_3 &= 1328 = 10100110000_2\\
    x_4 &= 1445 = 10110100101_2.
  \end{align*}
  Dakle, otvoreni tekst je $x = 1328$.
\end{solution}

\pagebreak

\question U Rabinovom kriptosustavu s parametrima
\begin{align*}
  (n, p, q) &= (2021, 43, 47),
\end{align*}
dešifrirajte šifrat $y = 917$. Poznato je da je otvoreni tekst prirodan broj $x < n$ kojem su zadnja četiri bita u binarnom zapisu međusobno jednaka.

\begin{solution}
  Najprije trebamo naći kvadratne korijene od $y$ modulo $p$ i modulo $q$. Budući da je $p \equiv q \equiv 3 \Mod{4}$, imamo
  \begin{align*}
    x &= \pm 917^{11} \equiv \pm 14^{10} \cdot 14 \equiv \pm 24^4 \cdot 24 \cdot 14 \equiv \pm 17^2 \cdot 35 \equiv \pm 10 \Mod{43},\\
    x &= \pm 917^{12} \equiv \pm 24^{12} \equiv \pm 12^6 \equiv \pm 3^3 \equiv \pm 27 \Mod{47}.
  \end{align*}
  Sada rješavamo ovaj sustav od 4 kongruencije malim kineskim teoremom. Dakle, trebamo naći $u$ i $v$ takve da je $47u + 43v = 1$. Te ćemo brojeve naći proširenim Euklidovim algoritmom.

  \begin{tabular}{|c|c|c|c|c|c|c|}
    \hline
    & & & & & &\\[-1em]
    $x$ & $y$ & $g$ & $u$ & $v$ & $w$ & $\left\lfloor \frac{g}{w} \right\rfloor$\\
    & & & & & &\\[-1em]
    \hline
    1 & 0 & 47 & 0 & 1 & 43 & 1\\
    0 & 1 & 43 & 1 & -1 & 4 & 10\\
    1 & -1 & 4 & -10 & 11 & 3 & 1\\
    -10 & 11 & 3 & 11 & -12 & 1 & 3\\
    11 & -12 & 1 & & & 0 &\\
    \hline
  \end{tabular}

  Dakle, $u = 11$ i $v = -12$. Sada su rješenja sustava kongruencija dana s $x \equiv 47 \cdot 11 \cdot (\pm 10) + 43 \cdot (-12) \cdot (\pm 27) \Mod{2021}$. Dakle, $x \equiv \pm 1343, \pm 913 \Mod{2773}$, tj. imamo
  \begin{align*}
    x_1 &= 1343 = 10100111111_2\\
    x_2 &= 913 = 1110010001_2\\
    x_3 &= 678 = 1010100110_2\\
    x_4 &= 1108 = 10001010100_2.
  \end{align*}
  Dakle, otvoreni tekst je $x = 1343$.
\end{solution}

\question Neka je u Diffie-Hellmanovom protokolu $G = \mathbb{Z}_p^*$, $p = 87671$, $g = 2$, $a = 1234$ i $b = 4321$. Odredite ključ $K = g^{ab}$.

\begin{solution}
  Budući da je $\varphi(87671) = 87670$ i $1234 \cdot 4321 \equiv 71914 \Mod{87670}$, imamo
  \begin{align*}
    K &= g^{a b} = 2^{1234 \cdot 4321} = 2^{71914}
  \end{align*}
  Nadalje, imamo:
  \begin{align*}
    2^{16} &= 65536, \quad 2^{32} = 52677, \quad 2^{64} = 79179 = -8492, \quad 2^{128} = 48502, \quad 2^{256} = 55732\\
    2^{512} &= 47636, \quad 2^{1024} = 3, \quad 2^{16384} = 3^{16} = 260, \quad 2^{65536} = 260^{4} = 84467\\\\
    2^{71914} &= 2^{65536} \cdot 2^{6144} \cdot 2^{128} \cdot 2^{64} \cdot 2^{32} \cdot 2^{10} = 84467 \cdot 3^6 \cdot 48502 \cdot (-8492) \cdot 52677 \cdot 1024\\
    &= 84467 \cdot 729 \cdot 48502 \cdot (-8492) \cdot 23583 = 31401 \cdot 48502 \cdot 61399 = 78361 \cdot 61399\\
    &= 77901
  \end{align*}
  Dakle, ključ je $K = 77901$.
\end{solution}

\question Zadan je ElGamalov kriptosustav s ključem $K = (p = 41, \alpha = 6, a = 10, \beta = 32)$. Dešifrirajte šifrat $(y_1, y_2) = (11, 21)$.

\begin{solution}
  Da bismo dešifrirali šifrat, potrebno je izračunati $y_2 \cdot (y_1^a)^{-1} \Mod{p}$. Dakle, prvo ćemo izračunati $y_1^{a} \Mod{p}$.
  \begin{align*}
    y_1^{a} \equiv 11^{10} \equiv (-2)^5 \equiv -32 \equiv 9 \Mod{41}.
  \end{align*}
  Sada trebamo izračunati inverz broja 9 modulo 41. To možemo napraviti proširenim Euklidovim algoritmom.

  \begin{tabular}{|c|c|c|c|c|c|c|}
    \hline
    & & & & & &\\[-1em]
    $x$ & $y$ & $g$ & $u$ & $v$ & $w$ & $\left\lfloor \frac{g}{w} \right\rfloor$\\
    & & & & & &\\[-1em]
    \hline
    1 & 0 & 41 & 0 & 1 & 9 & 4\\
    0 & 1 & 9 & 1 & -4 & 5 & 1\\
    1 & -4 & 5 & -1 & 5 & 4 & 1\\
    -1 & 5 & 4 & 2 & -9 & 1 & 4\\
    2 & -9 & 1 & & & 0 &\\
    \hline
  \end{tabular}

  Dakle, $41 \cdot (2) + 9 \cdot (-9) = 1$, tj. $9 \cdot (-9) \equiv 1 \Mod{41}$. Dakle, $y_1^{-a} \equiv 9^{-1} \equiv -9 \Mod{41}$. Sada možemo izračunati $y_2 \cdot (y_1)^{-a} \equiv 21 \cdot -9 \equiv 16 \Mod{41}$. Dakle, dešifrirani tekst je $16$.
\end{solution}

\question Neka je u ElGamalovom kriptosustavu $p = 1777$, $\alpha = 6$ i $a = 1009$.
\begin{parts}
  \part Šifrirajte otvoreni tekst $x = 1483$, uz pretpostavku da je jednokratni ključ $k = 701$.
  \part Dešifrirajte šifrat $(1664, 1031)$.
\end{parts}

\begin{solution}
  \begin{parts}
    \part Da bismo šifrirali otvoreni tekst, trebamo izračunati $e(x, k) = (\alpha^k \Mod{p}, x \cdot \beta^k \Mod{p})$. Dakle, imamo
    \begin{align*}
      \alpha^k &\equiv 6^{701} \equiv 6^{512} \cdot 6^{128} \cdot 6^{32} \cdot 6^{16} \cdot 6^{8} \cdot 6^{4} \cdot 6 \Mod{1777}\\
      &\equiv 1296^{128} \cdot 1296^{32} \cdot 1296^{8} \cdot 1296^{4} \cdot 1296^{2} \cdot 1296 \cdot 6 \Mod{1777}\\
      &\equiv 351^{64} \cdot 351^{16} \cdot 351^{4} \cdot 351^{2} \cdot 351 \cdot 668 \equiv 588^{32} \cdot 588^{8} \cdot 588^{2} \cdot 588 \cdot 1681 \Mod{1777}\\
      &\equiv 1006^{16} \cdot 1006^{4} \cdot 1006 \cdot 416 \equiv 923^{8} \cdot 923^{2} \cdot 901 \equiv 746^4 \cdot 746 \cdot 901 \Mod{1777}\\
      &\equiv 315^{2} \cdot 440 \equiv 1490 \cdot 440 \equiv 1664 \Mod{1777}.
    \end{align*}
    Sada možemo izračunati $\beta^k \equiv (\alpha^k)^a \Mod{p}$.
    \begin{align*}
      \beta^k &\equiv 1664^{1009} \equiv 1664^{512} \cdot 1664^{256} \cdot 1664^{128} \cdot 1664^{64} \cdot 1664^{32} \cdot 1664^{16} \cdot 1664 \Mod{1777}\\
      &\equiv 330^{256} \cdot 330^{128} \cdot 330^{64} \cdot 330^{32} \cdot 330^{16} \cdot 330^{8} \cdot 1664 \Mod{1777}\\
      &\equiv 503^{128} \cdot 503^{64} \cdot 503^{32} \cdot 503^{16} \cdot 503^{8} \cdot 503^{4} \cdot 1664 \Mod{1777}\\
      &\equiv 713^{32} \cdot 713^{16} \cdot 713^{8} \cdot 713^{4} \cdot 713^{2} \cdot 713 \cdot 1664 \Mod{1777}\\
      &\equiv 147^{16} \cdot 147^{8} \cdot 147^{4} \cdot 147^{2} \cdot 147 \cdot 1173 \equiv 285^{8} \cdot 285^{4} \cdot 285^{2} \cdot 285 \cdot 62 \Mod{1777}\\
      &\equiv 1260^{4} \cdot 1260^{2} \cdot 1260 \cdot 1677 \equiv 739^{2} \cdot 739 \cdot 167 \equiv 582 \cdot 800 \equiv 26 \Mod{1777}.
    \end{align*}
    Dakle, šifrat je $(1664, 1241)$.

    \part Da bismo dešifrirali šifrat, trebamo izračunati $y_2 \cdot (y_1^a)^{-1} \Mod{p}$. Dakle, prvo ćemo izračunati $y_1^{a} \Mod{p}$.
    \begin{align*}
      y_1^{a} \equiv 1664^{1009} \equiv 26 \Mod{1777}.
    \end{align*}
    Sada trebamo naći inverz broja 26 modulo 1777. To možemo napraviti proširenim Euklidovim algoritmom.

    \begin{tabular}{|c|c|c|c|c|c|c|}
      \hline
      & & & & & &\\[-1em]
      $x$ & $y$ & $g$ & $u$ & $v$ & $w$ & $\left\lfloor \frac{g}{w} \right\rfloor$\\
      & & & & & &\\[-1em]
      \hline
      1 & 0 & 1777 & 0 & 1 & 26 & 68\\
      0 & 1 & 26 & 1 & -68 & 9 & 2\\
      1 & -68 & 9 & -2 & 137 & 8 & 1\\
      -2 & 137 & 8 & 3 & -205 & 1 & 8\\
      3 & -205 & 1 & & & 0 &\\
      \hline
    \end{tabular}

    Dakle, $1777 \cdot 3 + 26 \cdot (-205) = 1$, tj. $26 \cdot (-205) \equiv 1 \Mod{1777}$. Dakle, $y_1^{-a} \equiv 26^{-1} \equiv -205 \Mod{1777}$. Sada možemo izračunati $y_2 \cdot (y_1)^{-a} \equiv 1031 \cdot -205 \equiv 108 \Mod{1777}$. Dakle, otvoreni tekst je $x = 108$.
  \end{parts}
\end{solution}

\question Zadan je Merkle-Hellmanov kriptosustav s ključem $K = (v, p, a, t)$ gdje je
\begin{align*}
  v &= (3, 6, 24, 48, 95, 187, 380, 760), \\
  p &= 1571, \ a = 111, \\
  t &= (333, 666, 1093, 615, 1119, 334, 1334, 1097).
\end{align*}
Dešifrirajte šifrat $y = 3379$.

\begin{solution}
  Najprije izračunamo inverz od $a$ modulo $p$. To možemo napraviti proširenim Euklidovim algoritmom.

  \begin{tabular}{|c|c|c|c|c|c|c|}
    \hline
    & & & & & &\\[-1em]
    $x$ & $y$ & $g$ & $u$ & $v$ & $w$ & $\left\lfloor \frac{g}{w} \right\rfloor$\\
    & & & & & &\\[-1em]
    \hline
    1 & 0 & 1571 & 0 & 1 & 111 & 14\\
    0 & 1 & 111 & 1 & -14 & 17 & 6\\
    1 & -14 & 17 & -6 & 85 & 9 & 1\\
    -6 & 85 & 9 & 7 & -99 & 8 & 1\\
    7 & -99 & 8 & -13 & 184 & 1 & 8\\
    -13 & 184 & 8 & & & 0 &\\
    \hline
  \end{tabular}

  Dakle, $a^{-1} \equiv 184 \Mod{1571}$. Sada računamo $z = y \cdot a^{-1} \Mod{p}$ i dobivamo $z = 1191$. Sada rješavamo superrastući problem ruksaka sa $v = (3, 6, 24, 48, 95, 187, 380, 760)$ i $V = 1191$. Budući da je $1191 = 760 + 380 + 48 + 3$, dobivamo $x = (1, 0, 0, 1, 0, 0, 1, 1)$.
\end{solution}

\end{questions}

\end{document}

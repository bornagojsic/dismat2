\documentclass{exam}
\usepackage[T1]{fontenc}
\usepackage{amsmath}
\usepackage{amssymb}
\usepackage{mathtools}
\usepackage{tikz}
\makeatletter
\newcommand\mathcircled[1]{%
  \mathpalette\@mathcircled{#1}%
}
\newcommand\@mathcircled[2]{%
  \tikz[baseline=(math.base)] \node[draw,circle,inner sep=1pt] (math) {$\m@th#1#2$};%
}
\usepackage[colorlinks=true, allcolors=blue]{hyperref}
\usepackage[makeroom]{cancel}
\usepackage[lmargin=71pt, tmargin=1.2in]{geometry}  %For centering solution box

\renewcommand{\thepartno}{\thequestion.\alph{partno}}
\renewcommand{\partlabel}{\alph{partno})}
\newcommand{\Mod}[1]{\ (\mathrm{mod}\ #1)}

\def \brojZadace {9}

\lhead{DISMAT 2 - Zadaća \brojZadace\\}
\rhead{Borna Gojšić\\}
% \chead{\hline} % Un-comment to draw line below header
\thispagestyle{empty}   %For removing header/footer from page 1

\begin{document}

\begingroup
\centering
\LARGE Diskretna matematika 2\\
\Large Zadaća \brojZadace\\
\large \today\\
\large Borna Gojšić\par
\endgroup
\rule{\textwidth}{0.4pt}
\pointsdroppedatright   %Self-explanatory
\printanswers
\renewcommand{\solutiontitle}{\noindent\textbf{Rj:}\enspace}   %Replace "Ans:" with starting keyword in solution box

\begin{questions}

\question Napišite tablice množenja i zbrajanja u polju $(\mathbb{Z}_7, +_7, \cdot_7)$.

\begin{solution}
  \begin{tabular}{|c|ccccccc|}
    \hline
    $+_7$ & 0 & 1 & 2 & 3 & 4 & 5 & 6 \\
    \hline
    0 & 0 & 1 & 2 & 3 & 4 & 5 & 6 \\
    1 & 1 & 2 & 3 & 4 & 5 & 6 & 0 \\
    2 & 2 & 3 & 4 & 5 & 6 & 0 & 1 \\
    3 & 3 & 4 & 5 & 6 & 0 & 1 & 2 \\
    4 & 4 & 5 & 6 & 0 & 1 & 2 & 3 \\
    5 & 5 & 6 & 0 & 1 & 2 & 3 & 4 \\
    6 & 6 & 0 & 1 & 2 & 3 & 4 & 5 \\
    \hline
  \end{tabular} \quad
  \begin{tabular}{|c|ccccccc|}
    \hline
    $\cdot_7$ & 0 & 1 & 2 & 3 & 4 & 5 & 6 \\
    \hline
    0 & 0 & 0 & 0 & 0 & 0 & 0 & 0 \\
    1 & 0 & 1 & 2 & 3 & 4 & 5 & 6 \\
    2 & 0 & 2 & 4 & 6 & 1 & 3 & 5 \\
    3 & 0 & 3 & 6 & 2 & 5 & 1 & 4 \\
    4 & 0 & 4 & 1 & 5 & 2 & 6 & 3 \\
    5 & 0 & 5 & 3 & 1 & 6 & 4 & 2 \\
    6 & 0 & 6 & 5 & 4 & 3 & 2 & 1 \\
    \hline
  \end{tabular}
\end{solution}

\question
\begin{parts}
  \part Izračunajte $3(2^3 + 5)^{-1} + 6$ u polju $(\mathbb{Z}_{7}, +_7, \cdot_7)$.
  \part Riješite jednadžbu $(2x + 9)(3x + 1)^{-1} = 7$ u polju $(\mathbb{Z}_{11}, +_{11}, \cdot_{11})$.
  \part Riješite jednadžbu $x^2 + 4(x^{-1} + 2x + 1) = 2$ u polju $(\mathbb{Z}_{5}, +_5, \cdot_5)$.
\end{parts}

\begin{solution}
  \begin{parts}
    \part $3(2^3 + 5)^{-1} + 6 = 3(1 + 5)^{-1} + 6 = 3 \cdot 6^{-1} + 6 = 3 + 6 = 2$.

    \part Znamo da $3x + 1 \neq 0$ pa imamo $3x \neq 10 = -1$, tj. $x \neq -3 = 8$.
    \begin{align*}
      (2x + 9)(3x + 1)^{-1} &= 7 \\
      2x + 9 &= 7(3x + 1) \\
      2x + 9 &= 9x + 7 \\
      9x - 2x &= 7 - 9 \\
      7x &= 9
    \end{align*}
    Očito je $x = 6$. Budući da je $(\mathbb{Z}_{11}, +_{11}, \cdot_{11})$ polje, rješenje je jedinstveno.

    \part Budući da imamo $x^{-1}$ u jednadžbi, znamo da je $x \neq 0$.
    \begin{align*}
      x^2 + 4(x^{-1} + 2x + 1) = 2 &\implies x^2 + 4x^{-1} + 3x + 4 = 2 \\
      x^2 - x^{-1} + 3x &= 3
    \end{align*}
    Sada možemo provjeriti sve elemente iz $\{1,2,3,4\}$.
    \begin{align*}
      1^2 - 1^{-1} + 3 \cdot 1 &= 1 - 1 + 3 = 3 \quad \checkmark \\
      2^2 - 2^{-1} + 3 \cdot 2 &= 4 - 3 + 1 = 2 \neq 3 \\
      3^2 - 3^{-1} + 3 \cdot 3 &= 4 - 2 + 4 = 1 \neq 3 \\
      4^2 - 4^{-1} + 3 \cdot 4 &= 1 - 4 + 2 = 4 \neq 3
    \end{align*}
    Dakle, jedino rješenje je $x = 1$.
  \end{parts}
\end{solution}

\question Na skupu racionalnih brojeva definirane su operacije $\triangle$ i $\square$ na sljedeći način:
\begin{align*}
  x \triangle y &= x + y + 1, \quad x \square y = xy + x + y.
\end{align*}
Dokažite da je $(\mathbb{Q}, \triangle, \square)$ prsten.

\begin{solution}
  Trebamo dokazati da je $(\mathbb{Q}, \triangle)$ abelova grupa i $(\mathbb{Q}, \square)$ polugrupa te da vrijedi distributivnost.
  \begin{enumerate}
    \item Zatvorenost operacije $\triangle$ slijedi iz zatvorenosti zbrajanja u $\mathbb{Q}$. Imamo
      \begin{align*}
        (x \triangle y) \triangle z &= (x + y + 1) + z + 1 = x + y + z + 2 = x + (y + z + 1) + 1 = x \triangle (y \triangle z)
      \end{align*}
      pa vrijedi asocijativnost. Neutralni element je $-1$ jer je
      \begin{align*}
        x \triangle -1 = x + (-1) + 1 = x = -1 + x + 1 = -1 \triangle x
      \end{align*}
      Inverz elementa $x$ je $-(x + 2)$ jer je
      \begin{align*}
        x \triangle (-(x + 2)) = x + (-(x + 2)) + 1 = -1 = -(x + 2) + x + 1 = -(x + 2) \triangle x
      \end{align*}
      Također, imamo komutativnost jer je
      \begin{align*}
        x \triangle y = x + y + 1 = y + x + 1 = y \triangle x
      \end{align*}
      Dakle, $(\mathbb{Q}, \triangle)$ je abelova grupa.

    \item Zatvorenost operacije $\square$ slijedi iz zatvorenosti množenja i zbrajanja u $\mathbb{Q}$. Imamo
      \begin{align*}
        (x \square y) \square z &= (xy + x + y)z + xy + x + y + z= xyz + xz + yz + xy + x + y + z\\
        x \square (y \square z) &= x(yz + y + z) + x + yz + y + z = xyz + xy + xz + x + yz + y + z
      \end{align*}
      pa je $(\mathbb{Q}, \square)$ polugrupa.

    \item Još trebamo provjeriti samo distributivnost. Imamo
      \begin{align*}
        x \square (y \triangle z) &= x \square (y + z + 1) = x(y + z + 1) + x + y + z + 1 = xy + x + y + xz + x + z + 1\\
        &= (x \square y) + (x \square z) + 1 = (x \square y) \triangle (x \square z)
      \end{align*}
      \begin{align*}
        (x \triangle y) \square z &= (x + y + 1) \square z = (x + y + 1)z + x + y + z + 1 = xz + x + z + yz + y + z + 1\\
        &= (x \square z) + (y \square z) + 1 = (x \square z) \triangle (y \square z)
      \end{align*}
  \end{enumerate}
  Dakle, $(\mathbb{Q}, \triangle, \square)$ je prsten.
\end{solution}

\pagebreak

\question Neka su $a \oplus b = a + b - 1$ i $a \otimes b = -\frac{ab}{2}$ binarne operacije na skupu $\mathbb{R}$. Ispitajte ima li $(\mathbb{R}, \oplus, \otimes)$ strukturu prstena.

\begin{solution}
  Trebamo dokazati da je $(\mathbb{R}, \oplus)$ abelova grupa i $(\mathbb{R}, \otimes)$ polugrupa te da vrijedi distributivnost.
  \begin{enumerate}
    \item Zatvorenost operacije $\oplus$ slijedi iz zatvorenosti zbrajanja u $\mathbb{R}$. Imamo
      \begin{align*}
        (a \oplus b) \oplus c &= (a + b - 1) + c - 1 = a + b + c - 2 = a + (b + c - 1) - 1 = a \oplus (b \oplus c)
      \end{align*}
      pa vrijedi asocijativnost. Neutralni element je $1$ jer je
      \begin{align*}
        a \oplus 1 = a + 1 - 1 = a = 1 + a - 1 = 1 \oplus a
      \end{align*}
      Inverz elementa $a$ je $2-a$ jer je
      \begin{align*}
        a \oplus (2 - a) = a + (2 - a) - 1 = 1 = 2 - a + a - 1 = (2 - a) \oplus a
      \end{align*}
      Također, imamo komutativnost jer je
      \begin{align*}
        a \oplus b = a + b - 1 = b + a - 1 = b \oplus a
      \end{align*}
      Dakle, $(\mathbb{R}, \oplus)$ je abelova grupa.

    \item Zatvorenost operacije $\otimes$ slijedi iz zatvorenosti množenja u $\mathbb{R}$. Imamo
      \begin{align*}
        (a \otimes b) \otimes c &= -\frac{ab}{2} \otimes c = -\frac{\frac{-ab}{2} \cdot c}{2} = \frac{abc}{4} = -\frac{a \cdot \frac{-bc}{2}}{2} = -\frac{a \cdot (b \otimes c)}{2} = a \otimes (b \otimes c)
      \end{align*}
      pa je $(\mathbb{R}, \otimes)$ polugrupa.

    \item Još trebamo provjeriti samo distributivnost. Imamo
      \begin{align*}
        a \otimes (b \oplus c) &= a \otimes (b + c - 1) = -\frac{a(b + c - 1)}{2} = -\frac{ab + ac - a}{2}
      \end{align*}
      ali je
      \begin{align*}
        (a \otimes b) \oplus (a \otimes c) &= -\frac{ab}{2} \oplus -\frac{ac}{2} = -\frac{ab}{2} - \frac{ac}{2} - 1 = -\frac{ab + ac + 2}{2}
      \end{align*}
      pa distributivnost ne vrijedi.
  \end{enumerate}
  Dakle, $(\mathbb{R}, \oplus, \otimes)$ nije prsten.
\end{solution}

\pagebreak

\question Dokažite da matrice oblika $
\begin{bmatrix}
  a & b \\
  2b & a
\end{bmatrix}$, gdje su $a$ i $b$ racionalni brojevi, uz uobičajeno zbrajanje i množenje matrica čine polje.

\begin{solution}
  Neka je $\mathcal{Q} = \left\{
    \begin{bmatrix}
      a & b\\
      2b & a
    \end{bmatrix}: a, b \in \mathbb{Q}
  \right\}$ Tada je
  \begin{align*}
    \begin{bmatrix}
      a & b\\
      2b & a
    \end{bmatrix} +
    \begin{bmatrix}
      c & d\\
      2d & c
    \end{bmatrix} =
    \begin{bmatrix}
      a + c & b + d\\
      2(b + d) & a + c
    \end{bmatrix} &\in \mathcal{Q}\\
    \begin{bmatrix}
      a & b\\
      2b & a
    \end{bmatrix} \cdot
    \begin{bmatrix}
      c & d\\
      2d & c
    \end{bmatrix} =
    \begin{bmatrix}
      ac + 2bd & ad + bc\\
      2(ad + bc) & ac + 2bd
    \end{bmatrix} &\in \mathcal{Q}
  \end{align*}
  Dakle, $\mathcal{Q}$ je zatvoren na zbrajanje i množenje matrica. Vrijedi asocijativnost zbrajanja i množenja iz asocijativnosti zbrajanja i množenja matrica. Neutralni element je $
  \begin{bmatrix}
    0 & 0\\
    0 & 0
  \end{bmatrix}$, a inverz od $A =
  \begin{bmatrix}
    a & b\\
    2b & a
  \end{bmatrix}$ je $A^{-1} =
  \begin{bmatrix}
    -a & -b\\
    -2b & -a
  \end{bmatrix}$. Distributivnost množenja nad zbrajanjem proizlazi iz distributvnosti matričnog množenja nad zbrajanjem. Dakle, $(\mathcal{Q}, +, \cdot)$ je prsten. Jedinica je očito $I$. Dakle, trebamo provjeriti da postoje inverzi u $(\mathcal{Q} \setminus \{0\}, \cdot)$. Imamo:
  \begin{align*}
    \begin{bmatrix}
      a & b\\
      2b & a
    \end{bmatrix} \cdot
    \begin{bmatrix}
      c & d\\
      2d & c
    \end{bmatrix} =
    \begin{bmatrix}
      ac + 2bd & ad + bc\\
      2(ad + bc) & ac + 2bd
    \end{bmatrix} =
    \begin{bmatrix}
      1 & 0\\
      0 & 1
    \end{bmatrix}
  \end{align*}
  Dakle,
  \begin{align*}
    ac + 2bd &= 1\\
    ad + bc &= 0 \implies c = -\frac{ad}{b}\\
    a\left(-\frac{ad}{b}\right) + 2bd &= 1 \implies d \left(-\frac{a^2}{b} + 2b \right) = 1\\
    d &= \frac{b}{2b^2-a^2}, \quad c = \frac{-a}{2b^2-a^2}
  \end{align*}
  Budući da je $a, b \neq 0$, to je $2b^2 - a^2 \neq 0$, tj. $\frac{1}{2b^2-a^2}
  \begin{bmatrix}
    -a & b\\
    2b & -a
  \end{bmatrix} \in \mathcal{Q}$. Dakle, $(\mathcal{Q}, +, \cdot)$ je tijelo. Za polje još trebamo dokazati komutativnost množenja:
  \begin{align*}
    \begin{bmatrix}
      c & d\\
      2d & c
    \end{bmatrix} \cdot
    \begin{bmatrix}
      a & b\\
      2b & a
    \end{bmatrix}
    =
    \begin{bmatrix}
      ac + 2bd & ad + bc\\
      2(ad + bc) & ac + 2bd
    \end{bmatrix} =
    \begin{bmatrix}
      a & b\\
      2b & a
    \end{bmatrix} \cdot
    \begin{bmatrix}
      c & d\\
      2d & c
    \end{bmatrix}
  \end{align*}
  Dakle, $(\mathcal{Q}, +, \cdot)$ je polje
\end{solution}

\pagebreak

\question Zadan je skup $T = \{a + b \sqrt{10} \ ; \ a,b \in \mathbb{Q}\}$.
\begin{parts}
  \part Dokažite da je skup $T$ polje uz uobičajeno zbrajanje i množenje realnih brojeva.
  \part Je li polje $T$ izomorfno polju racionalnih brojeva $(\mathbb{Q}, + , \cdot)$?
  \part Odredite inverz elementa $x = -3 + 2\sqrt{10}$ s obzirom na množenje.
\end{parts}

\begin{solution}
  \begin{parts}
    \part Neka su $x = a + b\sqrt{10}, y = c + d\sqrt{10} \in T$. Tada imamo
    \begin{align*}
      x + y &= (a + c) + (b + d)\sqrt{10} \in T\\
      x \cdot y &= (ac + 10bd) + (ad + bc)\sqrt{10} \in T
    \end{align*}
    Dakle, $T$ je zatvoren na zbrajanje i množenje. Vrijedi asocijativnost zbrajanja i množenja jer vrijedi asocijativnost zbrajanja i množenja realnih brojeva. Neutralni element zbrajanja je $0 \in \mathbb{R}$, a neutralni element množenja je $1 \in \mathbb{R}$. Inverz od $x = a + b\sqrt{10}$ s obzirom na zbrajanje je $-a - b\sqrt{10}$. Inverz od $x = a + b\sqrt{10}$ s obzirom na množenje je $\frac{a}{a^2 - 10b^2} - \frac{b}{a^2 - 10b^2}\sqrt{10}$ jer je
    \begin{align*}
      (a + b\sqrt{10})\left(\frac{a}{a^2 - 10b^2} - \frac{b}{a^2 - 10b^2}\sqrt{10}\right) &= \frac{a^2}{a^2 - 10b^2} - \frac{10b^2}{a^2 - 10b^2} = 1
    \end{align*}
    Distributivnost množenja nad zbrajanjem slijedi iz distributivnosti množenja realnih brojeva nad zbrajanjem. Dakle, $(T, +, \cdot)$ je polje.

    \part Neka je $\varphi: T \to \mathbb{Q}$ izomorfizam polja. Tada je $\varphi(0 + 0\sqrt{10}) = 0$ i $\varphi(1 + 0\sqrt{10}) = 1$ jer su to neutralni elementi zbrajanja i množenja. Tada imamo:
    \begin{align*}
      \varphi(2) = \varphi(1 + 1) = \varphi(1) + \varphi(1) = 2\\
      \varphi(4) = \varphi(2 + 2) = \varphi(2) + \varphi(2) = 4\\
      \varphi(8) = \varphi(4 + 4) = \varphi(4) + \varphi(4) = 8\\
      \varphi(10) = \varphi(8 + 2) = \varphi(8) + \varphi(2) = 10
    \end{align*}
    ali ako definiramo $q = \varphi(\sqrt{10})$, tada s druge strane imamo:
    \begin{align*}
      \varphi(10) = \varphi(\sqrt{10} \cdot \sqrt{10}) = \varphi(\sqrt{10}) \cdot \varphi(\sqrt{10}) = q^2 = 10
    \end{align*}
    Ali ne postoji $q \in \mathbb{Q}$ takav da je $q^2 = 10$. Dakle, $\varphi$ nije izomorfizam.

  \part Neka je $x = -3 + 2\sqrt{10}$. Tada je prema a) dijelu
  \begin{align*}
    x^{-1} = \frac{-3}{(-3)^2 - 10 \cdot 2^2} - \frac{2}{(-3)^2 - 10 \cdot 2^2}\sqrt{10} = \frac{3}{31} + \frac{2}{31}\sqrt{10}
  \end{align*}
\end{parts}
\end{solution}

\pagebreak

\question Dokažite da je skup $P = \{a + bi \ ; \ a,b \in \mathbb{Z}\}$ prsten uz uobičajeno zbrajanje i množenje kompleksnih brojeva. Je li skup $P$ polje? Obrazložite!

\begin{solution}
Neka su $z = a + bi, w = c + di \in P$. Tada imamo:
\begin{align*}
  z + w &= (a + c) + (b + d)i \in P\\
  z \cdot w &= (ac - bd) + (ad + bc)i \in P
\end{align*}
Dakle, $P$ je zatvoren na zbrajanje i množenje. Vrijedi asocijativnost zbrajanja i množenja jer vrijedi asocijativnost zbrajanja i množenja kompleksnih brojeva. Neutralni element zbrajanja je $0 \in \mathbb{C}$. Inverz od $z = a + bi$ s obzirom na zbrajanje je $-a - bi$. Komutativnost zbrajanja slijedi iz komutativnosti zbrajanja kompleksnih brojeva. Distributivnost množenja nad zbrajanjem slijedi iz distributivnosti množenja kompleksnih brojeva nad zbrajanjem. Dakle, $(P, +, \cdot)$ je prsten. Skup $P$ nije polje jer nema nužno inverz za množenje. Na primjer, $1 + i \in P$ nema inverz jer
\begin{align*}
  (1 + i)(a + bi) &= 1\\
  (a - b) + (a + b)i &= 1 \implies a = \frac{1}{2}, b = -\frac{1}{2}
\end{align*}
ali $a, b \notin \mathbb{Z}$.
\end{solution}

\question
\begin{parts}
\part Dokažite da je skup $P = \{a + bi \ ; \ a,b \in \mathbb{Q}\}$ polje uz uobičajeno zbrajanje i množenje kompleksnih brojeva.
\part Je li polje $P$ izomorfno standardnom polju racionalnih brojeva? Obrazložite!
\end{parts}

\begin{solution}
\begin{parts}
  \part Neka su $z = a + bi, w = c + di \in P$. Tada imamo:
  \begin{align*}
    z + w &= (a + c) + (b + d)i \in P\\
    z \cdot w &= (ac - bd) + (ad + bc)i \in P
  \end{align*}
  Dakle, $P$ je zatvoren na zbrajanje i množenje. Vrijedi asocijativnost zbrajanja i množenja jer vrijedi asocijativnost zbrajanja i množenja kompleksnih brojeva. Neutralni element zbrajanja je $0 \in \mathbb{C}$, a neutralni element množenja je $1 \in \mathbb{C}$. Inverz od $z = a + bi$ s obzirom na zbrajanje je $-a - bi$. Komutativnost zbrajanja i množenja slijedi iz komutativnosti zbrajanja kompleksnih brojeva. Distributivnost množenja nad zbrajanjem slijedi iz distributivnosti množenja kompleksnih brojeva nad zbrajanjem. Inverz od $z = a + bi$ s obzirom na množenje je $\frac{a}{a^2 + b^2} - \frac{b}{a^2 + b^2}i$ jer je
  \begin{align*}
    \left(a + bi\right) \left(\frac{a}{a^2 + b^2} - \frac{b}{a^2 + b^2}i\right) &= \frac{a^2}{a^2 + b^2} + \frac{b^2}{a^2 + b^2} = 1
  \end{align*}
  pa je $(P, +, \cdot)$ polje.

  \part Pretpostavimo da postoji izomorfizam $\varphi: P \to \mathbb{Q}$. Tada je $\varphi(1) = 1$ jer je $1$ neutralni element množenja. Također, $\varphi(1) = \varphi(-1 \cdot -1) = \varphi(-1) \cdot \varphi(-1) = 1$, pa je $\varphi(-1) = -1$. Ali, $\varphi(1) = \varphi(i^4) = \varphi(i)^4 = 1$ pa je $\varphi(i) = \pm 1$. Dakle, $\varphi$ nije pa nije ni izomorfizam. Dakle, $P$ nije izomorfno standardnom polju racionalnih brojeva.
\end{parts}
\end{solution}

\question Je li skup $T = \{a + b\sqrt[4]{2} \ ; \ a,b \in \mathbb{Q}\}$ prsten uz uobičajeno zbrajanje i množenje realnih brojeva?

\begin{solution}
Ne, jer $\sqrt{2} = \sqrt[4]{2} \cdot \sqrt[4]{2}$ nije u skupu $T$. Pretpostavimo suprotno, neka je $a + b \sqrt[4]{2} \in T$ i $a + b \sqrt[4]{2} = \sqrt{2}$. Tada imamo
\begin{align*}
  a + b \sqrt[4]{2} &= \sqrt{2} \\
  a - \sqrt{2} &= -b \sqrt[4]{2} \\
  a^2 - 2a\sqrt{2} + 2 &= b^2 \sqrt{2}\\
  a^2 + 2 &= (b^2 + 2a) \sqrt{2}
\end{align*}
Lijeva strana je racionalna, a desna strana je iracionalna, što je kontradikcija. Dakle, $\sqrt{2} \notin T$.
\end{solution}

\question Je li prsten $(\mathbb{Z}_{143}, +_{143}, \cdot_{143})$ integralna domena? Ukoliko jest, dokažite tu tvrdnju, a ukoliko nije navedite odgovarajući kontraprimjer.

\begin{solution}
Prsten $(\mathbb{Z}_{143}, +_{143}, \cdot_{143})$ nije integralna domena jer je $11 \cdot_{143} 13 = 0$.
\end{solution}

\question Dokažite da je skup
\begin{align*}
P = \{a + b \sqrt[3]{3} + c \sqrt[3]{9}: a, b, c \in \mathbb{Z}\}
\end{align*}
prsten uz uobičajeno zbrajanje i množenje realnih brojeva. Je li taj skup polje? Detaljno obrazložite!

\begin{solution}
Da bi $(P, +, \cdot)$ bio prsten, $(P, +)$ treba biti abelova grupa i $(P, \cdot)$ polugrupa.
\begin{align*}
  (a + b \sqrt[3]{3} + c \sqrt[3]{9}) + (d + e \sqrt[3]{3} + f \sqrt[3]{9}) &= (a + d) + (b + e) \sqrt[3]{3} + (c + f) \sqrt[3]{9} \in P\\
  (a + b \sqrt[3]{3} + c \sqrt[3]{9}) \cdot (d + e \sqrt[3]{3} + f \sqrt[3]{9}) &= (ad + 3bf + 9ce)\\ &+ (ae + bd + 3cf) \sqrt[3]{3} + (af + be + 3cd) \sqrt[3]{9} \in P
\end{align*}
Dakle $P$ je zatvoren na zbrajanje i množenje. Komutativnost i asocijativnost slijede iz asocijativnosti i komutativnosti zbrajanja i množenja realnih brojeva. Neutralni element zbrajanja je $0 \in P$, a neutralni element množenja je $1$. Inverz od elementa $x = a + b \sqrt[3]{3} + c \sqrt[3]{9}$ s obzirom na zbrajanje je $-a - b \sqrt[3]{3} - c \sqrt[3]{9}$. Distributivnost množenja nad zbrajanjem slijedi iz distributivnosti množenja realnih brojeva nad zbrajanjem. Dakle, $(P, +, \cdot)$ je prsten.
Neka je $x = \sqrt[3]{3}$ Tada bi njegov invez s obzirom na množenje bio $x^{-1} = a + b \sqrt[3]{3} + c \sqrt[3]{9}$ za koji je
\begin{align*}
  1 &= \sqrt[3]{3} (a + b\sqrt[3]{3} + c\sqrt[3]{9}) = 3c + a\sqrt[3]{3} + b\sqrt[3]{9} \implies a = 0, b = 0, c = \frac{1}{3}
\end{align*}
ali $c \notin \mathbb{Z}$, pa $(P, +, \cdot)$ nije polje.
\end{solution}

\end{questions}

\end{document}

\documentclass{exam}
\usepackage[T1]{fontenc}
\usepackage{amsmath}
\usepackage{amssymb}
\usepackage{mathtools}
\usepackage{tikz}
\makeatletter
\newcommand\mathcircled[1]{%
  \mathpalette\@mathcircled{#1}%
}
\newcommand\@mathcircled[2]{%
  \tikz[baseline=(math.base)] \node[draw,circle,inner sep=1pt] (math) {$\m@th#1#2$};%
}
\usepackage[colorlinks=true, allcolors=blue]{hyperref}
\usepackage[makeroom]{cancel}
\usepackage[lmargin=71pt, tmargin=1.2in]{geometry}  %For centering solution box

\renewcommand{\thepartno}{\thequestion.\alph{partno}}
\renewcommand{\partlabel}{\alph{partno})}
\newcommand{\Mod}[1]{\ (\mathrm{mod}\ #1)}

\def \brojZadace {2}

\lhead{DISMAT 2 - Zadaća \brojZadace\\}
\rhead{Borna Gojšić\\}
% \chead{\hline} % Un-comment to draw line below header
\thispagestyle{empty}   %For removing header/footer from page 1

\begin{document}

\begingroup
\centering
\LARGE Diskretna matematika 2\\
\Large Zadaća \brojZadace\\
\large \today\\
\large Borna Gojšić\par
\endgroup
\rule{\textwidth}{0.4pt}
\pointsdroppedatright   %Self-explanatory
\printanswers
\renewcommand{\solutiontitle}{\noindent\textbf{Rj:}\enspace}   %Replace "Ans:" with starting keyword in solution box

\begin{questions}

\question Dokažite da za svaki prirodan broj $n$ vrijedi
\begin{parts}
  \part $n^3 \equiv n \Mod{6}$
  \part $n^5 \equiv n \Mod{30}$
\end{parts}

\begin{solution}
  \begin{parts}
    \part Primijetimo da je $n^3 - n = n(n^2 - 1) = n(n - 1)(n + 1)$. $n - 1$, $n$ i $n + 1$ su 3 uzastopna prirodna broja. Dakle, bar 1 od njih je djeljiv s 2 i također bar 1 od njih je djeljiv s 3. Dakle, imamo $2 \mid n^3 - n$ i $3 \mid n^3 - n$, odnosno $6 \mid n^3 - n$, tj. $n^3 \equiv n \Mod{6}$.
    \part Primijetimo da je $n^5 - n = n(n^4 - 1) = n(n^2 - 1)(n^2 + 1) = n(n - 1)(n + 1)(n^2 + 1)$. Već smo dokazali da $6 \mid n(n + 1)(n - 1)$. Sada, ako $5 \nmid n, n-1, n+1$, onda imamo $n = 5k \pm 2$ za neki $k \in \mathbb{Z}$. Sada vidimo da je $n^2 + 1 = 25k^2 \pm 20k + 5$, odnosno $5 \mid n^2 + 1$. Dakle, $30 \mid n^5 - n$, tj. $n^5 \equiv n \Mod{30}$.
  \end{parts}
\end{solution}

\question
\begin{parts}
  \part Ako je $13x \equiv 13y \Mod{65}$, dokažite da je $x \equiv y \Mod{5}$. Vrijedi li obrat te tvrdnje?
  \part Ako je $a \equiv b \Mod{m}$, dokažite da je $a^2 \equiv b^2 \Mod{m}$. Vrijedi li obrat te tvrdnje? Obrazložite!
\end{parts}

\begin{solution}
  \begin{parts}
    \part Ako je $13x \equiv 13y \Mod{65}$, tada je $13(x - y) = 65k$ za neki $k \in \mathbb{Z}$. Iz toga slijedi da je $x - y = 5k$, odnosno $x \equiv y \Mod{5}$.
    \newline
    Ako je $x \equiv y \Mod{5}$, tada je $x - y = 5k$ za neki $k \in \mathbb{Z}$. Iz toga slijedi da je $13(x - y) = 65k$, odnosno $13x \equiv 13y \Mod{65}$.
    \part Ako je $a \equiv b \Mod{m}$, tada je $a - b = mk$ za neki $k \in \mathbb{Z}$. Iz toga slijedi da je $a^2 - b^2 = (a - b)(a + b) = mk(a + b)$, odnosno $a^2 \equiv b^2 \Mod{m}$.
    \newline
    Obrat te tvrdnje ne vrijedi. Na primjer, za $m = 5$, $a = 1 \equiv 1 \Mod{5}$ i $b = 4 \equiv -1 \Mod{5}$ imamo $a^2 \equiv 1 \equiv 1 \equiv b^2 \Mod{5}$, ali $a \equiv 1 \not\equiv 4 \equiv b \Mod{5}$.
  \end{parts}
\end{solution}

\question Riješite sljedeće kongruencije:
\begin{parts}
  \part $175x \equiv 252 \Mod{294}$
  \part $415x \equiv 15 \Mod{1115}$
  \part $238x \equiv 350 \Mod{420}$
\end{parts}

\pagebreak

\begin{solution}
  \begin{parts}
    \part Prvo ćemo Euklidovim algoritmom odrediti $d = \gcd(175, 294)$.
    \begin{tabular}{|c|c|c|}
      \hline
      & &\\[-1em]
      $a$ & $b$ & $\left\lfloor \frac{a}{b} \right\rfloor$\\
      & &\\[-1em]
      \hline
      294 & 175 & 1\\
      175 & 119 & 1\\
      119 & 56 & 2\\
      56 & 7 & 8\\
      7 & 0 &\\
      \hline
    \end{tabular}
    \vspace*{0.25cm}
    \newline
    Dakle, $d = 7$. $175:7 = 25$, $252:7 = 36$, $294:7 = 42$. Sada rješavamo kongruenciju $25x \equiv 36 \Mod{42}$.
    Koristit ćemo prošireni Euklidov algoritam za rješavanje kongruencije.
    \vspace*{0.25cm}
    \newline
    \begin{tabular}{|c|c|c|c||c|}
      \hline
      & & & &\\[-1em]
      $y$ & $g$ & $v$ & $w$ & $\left\lfloor \frac{a}{b} \right\rfloor$\\
      & & & &\\[-1em]
      \hline
      0 & 42 & 1 & 25 & 1\\
      1 & 25 & -1 & 17 & 1\\
      -1 & 17 & 2 & 8 & 2\\
      2 & 8 & -5 & 1 & 8\\
      -5 & 1 & & 0 &\\
      \hline
    \end{tabular}
    \vspace*{0.25cm}
    \newline
    Dakle, imamo $u \equiv -5 \Mod{42}$. Konačno, $x \equiv -5 \cdot 36 \equiv -180 \equiv 30 \Mod{42}$.
    Sada su sva rješenja početne kongruencije $x \equiv 30 + 42k \Mod{294}$ za $k \in \{0, 1, 2, 3, 4, 5, 6\}$.
    \part Prvo ćemo Euklidovim algoritmom odrediti $d = \gcd(415, 1115) = 5$. $415:5 = 83$, $15:5 = 3$, $1115:5 = 223$. Sada rješavamo kongruenciju $83x \equiv 3 \Mod{223}$.
    \vspace*{0.25cm}
    \newline
    \begin{tabular}{|c|c|c|c||c|}
      \hline
      & & & &\\[-1em]
      $y$ & $g$ & $v$ & $w$ & $\left\lfloor \frac{a}{b} \right\rfloor$\\
      & & & &\\[-1em]
      \hline
      0 & 223 & 1 & 83 & 2\\
      1 & 83 & -2 & 57 & 1\\
      -2 & 57 & 3 & 26 & 2\\
      3 & 26 & -8 & 5 & 5\\
      -8 & 5 & 43 & 1 & 5\\
      43 & 1 & & 0 &\\
      \hline
    \end{tabular}
    \vspace*{0.25cm}
    \newline
    Dakle, imamo $u \equiv 43 \Mod{223}$. Konačno, $x \equiv 43 \cdot 3 \equiv 129 \Mod{223}$. Sada su sva rješenja početne kongruencije $x \equiv 129 + 223k \Mod{1115}$ za $k \in \{0, 1, 2, 3, 4\}$.
    \pagebreak
    \part Euklidovim algoritmom dobijemo: $d = \gcd(238, 420) = 14$. Dakle, $238:14 = 17$, $350:14 = 25$, $420:14 = 30$, pa rješavamo kongruenciju $17x \equiv 25 \Mod{30}$.
    \vspace*{0.25cm}
    \newline
    \begin{tabular}{|c|c|c|c||c|}
      \hline
      & & & &\\[-1em]
      $y$ & $g$ & $v$ & $w$ & $\left\lfloor \frac{a}{b} \right\rfloor$\\
      & & & &\\[-1em]
      \hline
      0 & 30 & 1 & 17 & 1\\
      1 & 17 & -1 & 13 & 1\\
      -1 & 13 & 2 & 4 & 3\\
      2 & 4 & -7 & 1 & 4\\
      -7 & 1 & & 0 &\\
      \hline
    \end{tabular}
    \vspace*{0.25cm}
    \newline
    Dakle, imamo $u \equiv -7 \Mod{30}$. Konačno, $x \equiv -7 \cdot 25 \equiv -175 \equiv 5 \Mod{30}$. Sada su sva rješenja početne kongruencije $x \equiv 5 + 30k \Mod{420}$ za $k \in \{0, 1, 2, 3, \dots, 12, 13\}$.
  \end{parts}
\end{solution}

\pagebreak

\question
\begin{parts}
  \part Riješite kongruenciju $159x \equiv 66 \Mod{201}$.
  \part Odredite sve prirodne brojeve $n$ iz intervala $[1100, 1400]$ koji zadovoljavaju kongruenciju \newline $159n \equiv 66 \Mod{201}$.
  \part Odredite sve prirodne brojeve $m$ za koje vrijedi $159 \equiv 66 \Mod{m}$.
\end{parts}

\begin{solution}
  \begin{parts}
    \part Euklidovim algoritmom dobijemo: $d = \gcd(159, 201) = 3$. Dakle, $159:3 = 53$, $66:3 = 22$, $201:3 = 67$, pa rješavamo kongruenciju $53x \equiv 22 \Mod{67}$.
    \vspace*{0.25cm}
    \newline
    \begin{tabular}{|c|c|c|c||c|}
      \hline
      & & & &\\[-1em]
      $y$ & $g$ & $v$ & $w$ & $\left\lfloor \frac{a}{b} \right\rfloor$\\
      & & & &\\[-1em]
      \hline
      0 & 67 & 1 & 53 & 1\\
      1 & 53 & -1 & 14 & 3\\
      -1 & 14 & 4 & 11 & 1\\
      4 & 11 & -5 & 3 & 3\\
      -5 & 3 & 19 & 2 & 1\\
      19 & 2 & -24 & 1 & 2\\
      -24 & 1 & & 0 &\\
      \hline
    \end{tabular}
    \vspace*{0.25cm}
    \newline
    Dakle, imamo $u \equiv -24 \Mod{67}$. Konačno, $x \equiv -24 \cdot 22 \equiv -528 \equiv 8 \Mod{67}$. Sada su sva rješenja početne kongruencije $x \equiv 8, 75, 142 \Mod{201}$.
    \part Znamo da je $1100 \equiv 95 \Mod{201}$, pa je najmanji broj iz $[1100, 1400]$ koji zadovoljava kongruenciju $1100 + (142 - 95) = 1147$. Sada lako dobijemo sve brojeve iz intervala $[1100, 1400]$ koji zadovoljavaju kongruenciju: $1147, 1214, 1281, 1348$.
    \part $159 \equiv 66 \Mod{m}$ znači $m \mid 159 - 66 = 93$. Rastav $93$ na proste faktore je $93 = 3 \cdot 31$. Dakle, $m \in \{1, 3, 31, 93\}$.
  \end{parts}
\end{solution}

\question Riješite sljedeće sustave kongruencija:
\begin{parts}
  \part $x \equiv 7 \Mod{17}$, $x \equiv 18 \Mod{31}$, $x \equiv 33 \Mod{37}$
  \part $x \equiv 2 \Mod{5}$, $x \equiv 1 \Mod{6}$, $x \equiv 4 \Mod{11}$, $x \equiv 5 \Mod{17}$
  \part $5x \equiv 3 \Mod{7}$, $16x \equiv 7 \Mod{17}$, $25x \equiv 2 \Mod{37}$.
\end{parts}

\begin{solution}
  \begin{parts}
    \part Imamo $m = 17 \cdot 31 \cdot 37 = 19499$ i $x_0 = 1147x_1 + 629x_2 + 527x_3$. Dakle, imamo sljedeći sustav:
    \[
      1147x_1 \equiv 7 \Mod{17}, \quad 629x_2 \equiv 18 \Mod{31}, \quad 527x_3 \equiv 33 \Mod{37}
    \]
    \[
      8x_1 \equiv 7 \Mod{17}, \quad 9x_2 \equiv 18 \Mod{31}, \quad 9x_3 \equiv 33 \Mod{37}
    \]
    Dakle, imamo $x_1 = 3$, $x_2 = 2$ i $x_3 = 16$, $x_0 = 13131$ i $x \equiv 13131 \Mod{19499}$.
    \part Imamo $m = 5 \cdot 6 \cdot 11 \cdot 17 = 5610$ i $x_0 = 1122x_1 + 935x_2 + 510x_3 + 330x_4$. Dakle, imamo sljedeći sustav:
    \[
      1122x_1 \equiv 2 \Mod{5}, \quad 935x_2 \equiv 1 \Mod{6}, \quad 510x_3 \equiv 4 \Mod{11}, \quad 330x_4 \equiv 5 \Mod{17}
    \]
    \[
      2x_1 \equiv 2 \Mod{5}, \quad 5x_2 \equiv 1 \Mod{6}, \quad 4x_3 \equiv 4 \Mod{11}, \quad 7x_4 \equiv 5 \Mod{17}
    \]
    Dakle, imamo $x_1 = 1$, $x_2 = -1$, $x_3 = 1$ i $x_4 = 8$, $x_0 = 3337$ i $x \equiv 3337 \Mod{5610}$.
    \part Ovaj sustav ćemo prvo dovesti u oblik $x \equiv a \Mod{m}$ što možemo jer su svi moduli prosti brojevi pa postoje multiplikativni inverzi modulo $m_j$.
    \[
      5x \cdot 3 \equiv 3 \cdot 3 \Mod{7}, \quad 16x \cdot (-1) \equiv 7 \cdot (-1) \Mod{17}, \quad 25x \cdot 3 \equiv 2 \cdot 3 \Mod{37}
    \]
    \[
      x \equiv 2 \Mod{7}, \quad x \equiv 10 \Mod{17}, \quad x \equiv 6 \Mod{37}
    \]
    Sada imamo $m = 7 \cdot 17 \cdot 37 = 4403$ i $x_0 = 629x_1 + 259x_2 + 119x_3$. Dakle, imamo sljedeći sustav:
    \[
      629x_1 \equiv 2 \Mod{7}, \quad 259x_2 \equiv 10 \Mod{17}, \quad 119x_3 \equiv 6 \Mod{37}
    \]
    \[
      -x_1 \equiv 2 \Mod{7}, \quad 4x_2 \equiv 10 \Mod{17}, \quad 8x_3 \equiv 6 \Mod{37}
    \]
    Dakle, $x_1 = -2$, $x_2 = 11$, $x_3 = 10$, $x_0 = 2781$ i $x \equiv 2781 \Mod{4403}$.
  \end{parts}
\end{solution}

\question Riješite sljedeće sustave kongruencija:
\begin{parts}
  \part $x \equiv 10 \Mod{15}$, $x \equiv 19 \Mod{21}$, $x \equiv 25 \Mod{60}$
  \part $x \equiv 13 \Mod{16}$, $x \equiv 5 \Mod{24}$, $x \equiv 8 \Mod{27}$, $x \equiv 2 \Mod{5}$.
\end{parts}
\underline{Napomena}: Uočite da moduli nisu u parovima relativno prosti.

\begin{solution}
  \begin{parts}
    \part Kongruencije rastavljamo na kongruencije potencija prostih modula.
    \begin{align*}
      x &\equiv 10 \Mod{15} \implies x \equiv 10 \Mod{3}, \quad x \equiv 10 \Mod{5}\\
      x &\equiv 19 \Mod{21} \implies x \equiv 19 \Mod{3}, \quad x \equiv 19 \Mod{7}\\
      x &\equiv 25 \Mod{60} \implies x \equiv 25 \Mod{3}, \quad x \equiv 25 \Mod{4}, \quad x \equiv 25 \Mod{5}
    \end{align*}
    Dakle, imamo:
    \begin{align*}
      x \equiv 10 \Mod{3}, \quad x \equiv 19 \Mod{3}, \quad x \equiv 25 \Mod{3} &\implies x \equiv 1 \Mod{3}\\
      x \equiv 25 \Mod{4} &\implies x \equiv 1 \Mod{4}\\
      x \equiv 10 \Mod{5}, \quad x \equiv 25 \Mod{5} &\implies x \equiv 0 \Mod{5}\\
      x \equiv 19 \Mod{7} &\implies x \equiv 5 \Mod{7}
    \end{align*}
    Sada možemo primijeniti kineski teorem o ostacima. Imamo $m = 3 \cdot 4 \cdot 5 \cdot 7 = 420$ i $x_0 = 140x_1 + 105x_2 + 84x_3 + 60x_4$. Dakle, imamo sljedeći sustav:
    \[
      140x_1 \equiv 1 \Mod{3}, \quad 105x_2 \equiv 1 \Mod{4}, \quad 84x_3 \equiv 0 \Mod{5}, \quad 60x_4 \equiv 5 \Mod{7}
    \]
    \[
      2x_1 \equiv 1 \Mod{3}, \quad x_2 \equiv 1 \Mod{4}, \quad 4x_3 \equiv 0 \Mod{5}, \quad 4x_4 \equiv 5 \Mod{7}
    \]
    Dakle, imamo $x_1 = 2$, $x_2 = 1$, $x_3 = 0$, $x_4 = 3$, $x_0 = 565$ i $x \equiv 565 \equiv 145 \Mod{420}$.
    \part
    \begin{align*}
      x &\equiv 13 \Mod{16}\\
      x &\equiv 5 \Mod{24} \implies x \equiv 5 \Mod{3}, \quad x \equiv 5 \Mod{8}\\
      x &\equiv 8 \Mod{27}\\
      x &\equiv 2 \Mod{5}\\
    \end{align*}
    Dakle, imamo:
    \begin{align*}
      x \equiv 13 \Mod{16}, \quad x \equiv 5 \Mod{8} \implies &x \equiv 13 \Mod{16}\\
      x \equiv 5 \Mod{3}, \quad x \equiv 8 \Mod{27} \implies &x \equiv 8 \Mod{27}\\
      &x \equiv 2 \Mod{5}
    \end{align*}
    Sada možemo primijeniti kineski teorem o ostacima. Imamo $m = 16 \cdot 27 \cdot 5 = 2160$ i $x_0 = 135x_1 + 80x_2 + 432x_3$. Dakle, imamo sljedeći sustav:
    \[
      135x_1 \equiv 13 \Mod{16}, \quad 80x_2 \equiv 8 \Mod{27}, \quad 432x_3 \equiv 2 \Mod{5}
    \]
    \[
      7x_1 \equiv 13 \Mod{16}, \quad -x_2 \equiv 8 \Mod{27}, \quad 2x_3 \equiv 2 \Mod{5}
    \]
    Dakle, $x_1 = 11$, $x_2 = -8$, $x_3 = 1$, $x_0 = 1277$ i $x \equiv 1277 \Mod{2160}$.
  \end{parts}
\end{solution}

\question Odredite najmanji prirodan broj koji pri dijeljenju s brojevima 41, 42 i 43 daje ostatke 1, 2 i 3 (u tom redoslijedu).

\begin{solution}
  Ovo je ekvivalentno rješavanju sustava kongruencija:
  \[
    x \equiv 1 \Mod{41}, \quad x \equiv 2 \Mod{42}, \quad x \equiv 3 \Mod{43}
  \]
  Budući da su 41 i 43 prosti brojevi, možemo koristiti kineski teorem o ostacima. Imamo $m = 74046$ i $x_0 = 1806x_1 + 1763x_2 + 1722x_3$. Dakle, imamo sljedeći sustav:
  \[
    1806x_1 \equiv 1 \Mod{41}, \quad 1763x_2 \equiv 2 \Mod{42}, \quad 1722x_3 \equiv 3 \Mod{43}
  \]
  \[
    2x_1 \equiv 1 \Mod{41}, \quad 41x_2 \equiv 2 \Mod{42}, \quad 2x_3 \equiv 3 \Mod{43}
  \]
  Dakle, $x_1 = 21$, $x_2 = -2$, $x_3 = 23$, $x_0 = 74006$ i $x \equiv 74006 \Mod{74046}$. Budući da je $74006 < 74046$, najmanji prirodan broj koji zadovoljava uvjete je 74006.
\end{solution}

\question
\begin{parts}
  \part Odredite najmanji prirodan broj $n$ takav da $3^2 \mid n$, $4^2 \mid n + 1$ i $5^2 \mid n + 2$.
  \part Postoji li prirodan broj $n$ takav da $2^2 \mid n$, $3^2 \mid n + 1$ i $4^2 \mid n + 2$? Obrazložite!
\end{parts}

\begin{solution}
  \begin{parts}
    \part Budući da su svi moduli relativno prosti u parovima, možemo primijeniti kineski teorem o ostacima. Imamo $m = 3^2 \cdot 4^2 \cdot 5^2 = 3600$ i $x_0 = 400x_1 + 225x_2 + 144x_3$. Dakle, imamo sljedeći sustav:
    \[
      400x_1 \equiv 0 \Mod{9}, \quad 225x_2 \equiv -1 \Mod{16}, \quad 144x_3 \equiv -2 \Mod{25}
    \]
    \[
      4x_1 \equiv 0 \Mod{9}, \quad x_2 \equiv -1 \Mod{16}, \quad 19x_3 \equiv -2 \Mod{25}
    \]
    Dakle, $x_1 = 0$, $x_2 = -1$ i $x_3 = 17$, $x_0 = 2223$ i budući da je $2223 < 3600$, najmanji takav $n$ je 2223.
    \part Iz prve kongruencije imamo $n = 4k$, a iz zadnje $n = 16l - 2$. Dakle $4k = 16l - 2$, odnosno $2k = 8l - 1$. Lijeva strana je paran broj, a desna je neparan, pa nema rješenja.
  \end{parts}
\end{solution}

\question Neka je $p$ prost broj.
\begin{parts}
  \part Dokažite da je $\binom{p}{k} \equiv 0 \Mod{p}$ za $k \in \{1, 2, \dots, p - 1\}$.
  \part Dokažite da za svaki cijeli broj $n$ vrijedi $(n + 1)^p \equiv n^p + 1 \Mod{p}$.
\end{parts}

\begin{solution}
  \begin{parts}
    \part $\binom{p}{k} = \frac{p!}{k!(p - k)!}$. Ako je $k \in \{1, 2, \dots, p - 1\}$, tada $p \mid p!$, $p \nmid k!$ i $p \nmid (p - k)!$. Dakle, $p \mid \binom{p}{k}$.
    \part
    \begin{align*}
      (n + 1)^p &= \sum_{k = 0}^{p} \binom{p}{k} n^k\\
      &= n^p + \sum_{k = 1}^{p - 1} \binom{p}{k} n^k + 1\\
      &\equiv n^p + 1 \Mod{p}
    \end{align*}
  \end{parts}
\end{solution}

\end{questions}

\end{document}
